\documentclass{beamer}
\usetheme{Warsaw}
\usepackage{nhtvslides}
\usepackage{graphicx}
\usepackage{listings}
\lstset{language=CAML,
basicstyle=\ttfamily\footnotesize,
frame=shadowbox,
breaklines=true}
\usepackage[utf8]{inputenc}

\title{Game development in Casanova}

\author{Giuseppe Maggiore}

\institute{Università Ca' Foscari - Venezia, Italy \\ NHTV University - Breda, Netherlands}

\date{}

\begin{document}
\maketitle

\begin{frame}{Table of contents}
\tableofcontents
\end{frame}

\section{Introduction}
\begin{slide}{Motivation}{Game development}{
\item A huge entertainment industry (recently surpassed movies and music)
\item A social innovator (media and interaction at the same time)
\item A driver for artistic and technological innovation
%[[ game playing, industry size ]]
}\end{slide}

\begin{slide}{Motivation}{Pressure on game developers}{
\item Making a single game costs between thousands and millions of Euro's
\item Tens of artists, programmers, designers, and managers
%[[ game costs ]]
}\end{slide}

\begin{slide}{Motivation}{Pressure on big game developers}{
\item AAA games, the biggest titles, cost a lot to develop
\item Many games do not turn a profit
\item When a formula works, it gets repeated a lot
\item Less innovation and experimentation
}\end{slide}

\begin{slide}{Motivation}{An example AAA game}{
\item GTA IV
\item 100,000,000 \$
\item 1000 people
\item 3.5 years
%[[ GTA IV ]]
}\end{slide}

\begin{slide}{Motivation}{Smaller developers}{
\item There are smaller game developers
\item Indie, Research, and Serious
%[[ i, r, s games ]]
}\end{slide}

\begin{slide}{Motivation}{Pressure on smaller developers}{
\item These have far less resources...
\item ...but they innovate and have the potential for significant impact
}\end{slide}

\begin{slide}{Motivation}{An example AA game}{
\item Journey
\item A few million dollars
\item Around twenty between developers, artists, designers, etc.
%[[ Journey ]]
}\end{slide}

\begin{slide}{Motivation}{Motivation of our research}{
\item The significance of games, especially smaller ones
\item The financial pressure on smaller developers
\item A study of disciplined game development to reduce costs and effort, and hopefully help the industry reach another level of creativity and experimentation
%[[ Some motivational futuristic image on research ]]
}\end{slide}

\section{Technical introduction}
\begin{slide}{Technical introduction}{How games are made}{
\item We start by studying the structure of games
\item We will then find general models for game architectures
\item There are many patterns that can be found, but we start with the basics
}\end{slide}

\begin{slide}{Technical introduction}{High level model of a game}{
\item A game is a \textit{hybrid dynamic system}
\item A ``state'' that stores information about the system
\item Differential equations that model ``flow''
\item Control graphs that model ``jumps''
\item A numerical method to approximate DE and CG
%[[ HYBRID SYSTEM ]]
}\end{slide}

\begin{slide}{Technical introduction}{Typical architecture of a game}{
\item Data structures to model the state
\item A game loop that:
\begin{itemize}
\item applies the numerical integrator (to a small $\delta t$)
\item applies the transitions of the control graph
\item invokes the draw function
\end{itemize}
}\end{slide}


\begin{slide}{Technical introduction}{Traditional solutions and their shortcomings}{
\item there are two traditional solutions that the game development industry uses
\begin{itemize}
\item libraries that collect many common data-types and operations such as vectors, matrices, scene graphs, timers, etc.
\item systems (engines) that are essentially heavily parametrized games where appearance and parts of the game logic may be modified
\end{itemize}
\item unfortunately, both represent a trade-off between expressive power and development speed and ease
}\end{slide}


\begin{slide}{Technical introduction}{Shortcomings of libraries}{
\item DirectX, OpenGL, XNA, etc.
\item emphasize small pieces
\item high expressive power: you write your whole game and can rebuild whole parts of the library
\item much slower development, more bugs
\item games need something more pervasive than just a series of handy bits
}\end{slide}


\begin{slide}{Technical introduction}{Shortcomings of libraries}{
\item games need something more pervasive than just a series of handy bits
\begin{itemize}
\item this pervasiveness is also at odds with hierarchies of virtual classes
\item having to access every game entity through a collection of virtual methods is too slow
\item up to 30\% penalty measured by Sony PS3 developers against a data-oriented approach
\item this makes interactivity harder to reach
\end{itemize}
}\end{slide}


\begin{slide}{Technical introduction}{Shortcomings of systems}{
\item Unity, Game Maker, Unreal Engine
\item emphasize a whole game
\item leave as much as possible pluggable
\item visual interfaces plus scripting
\item very fast development
\item limited expressive power: if something is not left open to change, your game must work around it or just use it
%[[ UNITY, UNREAL ENGINE ]]
}\end{slide}

\begin{slide}{Technical introduction}{Advantages of designing a DSL}{
\item Inform, Simula
\item search for the ``ideal'' tension between brevity, clarity, and expressive power
\item good syntax: short and clear programs
\item good and expressive static analysis tools: less debugging
\item good semantics: high expressive power
\item automated optimization: higher performance
}\end{slide}


\section{Casanova}
\begin{slide}{Technical introduction}{Design directions for the Casanova language}{
\item declarative when possible (rules, drawing)
\item imperative when needed (coroutines)
\item cheap granular concurrency (coroutines)
\item fast (automated optimizations: multi-threading, query optimization, batching, update frequencies)
}\end{slide}


\begin{slide}{Technical introduction}{Structure of a Casanova game}{
\item world as a series of entities
\item rules define the transformation of a field of an entity as time progresses (integration step)
\item drawable entities found in the world are drawn automatically
\item coroutines are stepped over right after every update
}\end{slide}


\begin{slide}{Technical introduction}{Semantics of Casanova}{
\item types and rules are straightforwardly translated into F\# with double buffering
\item drawing is translated into a series of calls to MonoGame primitives
\item scripts are translated into a custom monad for soft-threads
}\end{slide}


\begin{slide}{Technical introduction}{Three implementations}{
\item naïve reflection
\item cached reflection
\item full-blown assembly generator
}\end{slide}


\begin{slide}{Technical introduction}{Naïve reflection}{
\item very straightforward way to test the language
\item excessively slow: impossible to use
\item reflection operations are the most expensive: \texttt{GetType}, \texttt{GetMethod}, etc.
}\end{slide}


\begin{frame}[fragile]{Naïve reflection}
\begin{lstlisting}
let traverse (e:obj) =
  let e_t = e.GetType()
  match e_t with
  | Record(fields) ->
    apply_rules e
    for f in fields do
      let e_f = f.GetGetMethod().Invoke(e)
      do traverse e_f
  | Array ->
    let get_length = e_t.GetMethod("get_Length")
    let get_item   = e_t.GetMethod("get_Item")
    for i = 0 to get_length.Invoke(e) - 1 do
      do traverse(get_item(e,i))
  | ...
\end{lstlisting}
\end{frame}

\begin{slide}{Technical introduction}{Cached reflection}{
\item uses continuation passing style to cache the reflection calls
\item much faster: has been the reference implementation for a long time
\item continuation operators are the most expensive: \texttt{cont.Bind}, \texttt{cont.Return}
}\end{slide}


\begin{frame}[fragile]{Cached reflection}
\begin{lstlisting}
let traverse (e_t:Type) (get_e:obj->obj) =
  cont{
    match e_t with
    | Record(fields) ->
      do! apply_rules(get_e,e_t)
      for f in fields do
        let e_f = f.GetGetMethod()
        do! traverse f.Type (get_e >> e_f)
    | Array(a) ->
      let get_length = e_t.GetMethod("get_Length")
      let get_item   = e_t.GetMethod("get_Item")
      do! for_cont (fun () -> get_e >> get_length)
                   (fun i -> traverse(get_e >> get_item(e,i))
    | ...
  }
\end{lstlisting}
\end{frame}

\begin{slide}{Technical introduction}{Emit}{
\item generate the actual assembly operations needed to traverse the world
\item the actual compiler for Casanova
\item no traversal overhead, no casts from object, no reflection at runtime
\item aggressive inlining
\item sequence iterators are now the most expensive: \texttt{seq.Bind}, \texttt{seq.Concat}
}\end{slide}


\begin{frame}[fragile]{Emit}
\begin{lstlisting}
let traverse (e_t:Type) =
    match e_t with
    | Array(a) ->
      let i = il.DeclareLocal(typeof<int>)
      let get_length = e_t.GetMethod("get_Length")
      let get_item   = e_t.GetMethod("get_Item")
      do il.Emit(DUP)
      do il.Emit(CALL, get_length)
      do il.Emit(SET_LOC, i)
      let loop = il.MarkLabel()
      let end  = il.DeclareLabel()
      // decr i by one
      // jump to end if negative index
      // load i-th element
      do traverse a
      do il.Emit(JMP, loop)
    | ...
\end{lstlisting}
\end{frame}

\begin{frame}{Technical introduction}
\begin{block}{Performance}
\begin{table}
\center
\begin{tabular}{| c | c | c |}
\hline
City no CPS & City CPS & City Emit \\
\hline
0.2 & 30 & 120 \\
\hline
\end{tabular}
\caption{Samples speed}
\end{table}
\end{block}
\end{frame}

\section{Closing}
\begin{slide}{Closing}{Evaluation}{
\item We evaluated Casanova across three dimensions:
\begin{itemize}
\item The central analytical evaluation is: ``Can games be built with it?''
\item Usability evaluations: can students or others make games with it?
\item Implicit performance evaluations: these games run interactively on COTS hardware
\item Explicit performance and LOC evaluations: Casanova vs other languages
\end{itemize}
}\end{slide}


\begin{slide}{Closing}{Can games be built with it?}{
\item Most certainly:
\begin{itemize}
\item Asteroid shooter
\item Small RTS
\item Bigger RTS
\item and more...
\end{itemize}
\item DEMO
}\end{slide}


\begin{slide}{Closing}{Can students or others make games with it?}{
\item Yes
\begin{itemize}
\item Longer exposure, previous knowledge of F\#: Tetris, Pac-Man
\item Very short exposure, previous knowledge of C++: Rocket, Platformer
\end{itemize}
\item DEMO
}\end{slide}


\begin{slide}{Motivation}{Galaxy Wars}{
\item An actual commercial game made in Casanova
\item Published this week
\item DEMO
}\end{slide}


\begin{slide}{Motivation}{Casanova vs other languages}{
\item We compared implementations of small game snippets
\item Mostly centred on the heaviest aspect of computation: scripts
}\end{slide}


\begin{frame}[fragile]{Casanova vs other languages}
\begin{table}
\center
\begin{tabular}{| l | c | c | c |}
\hline
Language & Test 1 & Test 2 & Test 3 \\
\hline
Casanova & 21 & 21 & 35 \\
Python & 24 & 29 & 48 \\
LUA & 30 & 39 & 52 \\
C\# & 51 & 58 & 59 \\
\hline
\end{tabular}
\caption{Samples length}
\end{table}
\end{frame}

\begin{frame}{Casanova vs other languages}
\begin{table}
\center
\begin{tabular}{| l | c | c | c |}
\hline
Language & Test 1 & Test 2 & Test 3 \\
\hline
Casanova & 7.6 & 5.8 & 4.0 \\
Python & 1.1 & 1.1 & 0.9 \\
LUA & 1.5 & 1.4 & 0.8 \\
C\# & 7.1 & 4.2 & 4.1 \\
\hline
\end{tabular}
\caption{Samples speed}
\end{table}
\end{frame}

\section{Extensions \& Conclusions}
\begin{slide}{Motivation}{Extensions}{
\item Declarative actions and their optimization
\item Menus and nesting of visuals for complex layouts
}\end{slide}


\begin{slide}{Motivation}{Future developments}{
\item Audio support
\item Integration with a better back-end (Unity 3D)
\item Mobile support (Android, iOS, Win RT)
}\end{slide}


\begin{slide}{Motivation}{Conclusions}{
\item Game development is big, but expensive; this limits its impact on society by reducing creativity and less obvious applications
\item Traditional systems for making games fall short of the mark in either expressive power or development time
\item Casanova was created to look for the ``sweet spot'' between expression and effort
\item The results so far are very encouraging; we are using it heavily in-house and have just started promoting it
}\end{slide}


\begin{frame}{This is it}
\begin{block}{Thank you!}
Questions?
\end{block}
\end{frame}

\end{document}
