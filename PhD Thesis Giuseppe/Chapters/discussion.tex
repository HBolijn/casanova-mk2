In this section we sum up the content of this work, with particular focus on: \textit{(i)} how Casanova answers the original questions of this thesis; \textit{(ii)} what are the limitations of Casanova that may be solved with small extensions; \textit{(iii)} what are the limitations of Casanova that may be solved with a major design effort or not at all.

\section{Original research questions}
The research questions studied by this work are listed in Chapter \ref{chap:introduction}, but we sum them up here again:

\begin{tabular}{ | l | p{6cm} | }
\hline
\textbf{requirements} & what are the general requirements common across development of most games? \\
\hline
\textbf{exploration} & what are the most representative game development systems and languages? \\
\hline
\textbf{design} & what is a general and simple to use programming language that fulfills the game development requirements? \\
\hline
\textbf{evaluation} & Does such a language work for making games in practice? How does such it compare to other game development systems? \\
\hline
\end{tabular}


We have identified a set of requirements by observing which are the most common tasks performed by game developers while making a game. These are tasks that allow the simulation of a virtual world that is updated in real-time, steered by the user input, while providing a real-time visual representation of the virtual world. We have identified these requirements through our experience with game development, by studying the architectures of multiple game development tools and libraries, and by a survey of the fundamental literature of game development which all mentions these constructs. Even though we believe to have identified a set of universal, core requirements of games, we are also certain that further important requirements exist that we are not considering. We also believe that the requirements we consider here are the first that should be tackled, since considering more exotic issues would require that these are solved first.

A preliminary study of the available game development systems yielded dozens of powerful systems, engines, and frameworks, plus various languages. As game development is a complex activity, so game development tools are large and difficult to learn. To narrow our focus we started by restricting ourselves to only those tools that are aimed at making the same kind of games that Casanova supports, namely serious, research, and indie games. Secondly, we picked the three systems that seem the most popular in terms of available books, tutorials, documentation, and sample projects.
Choosing game development languages on the other hand was simpler, since there are not many available. We picked three languages that exemplified that language design does not need to be a derivative effort that yields yet another imperative language with objects; we picked languages that show provocative new semantic and syntactic structures.

We designed Casanova around the identified requirements for games, by choosing the design philosophy of less-is-more. We tried to find the smallest set of orthogonal syntactic primitives that would allow the expression of the desired semantics. We built Casanova semantics in order to express game concepts of input, drawing, time flow, state machines, and few other primitives. We used our type system to support useful correctness enforcers such as dimensional analysis. We built an implementation that optimizes run-time by taking advantage of the semantics of the game, for example by adding multi-threading automatically where it is safe to do so.

We evaluated Casanova by showing how actual game snippets can be produced in it, and by identifying exemplar game development tasks such as defining an avatar, interacting and moving in a virtual environment with entities representing objects and monsters, and so on. We then proceeded to a comparison of Casanova with existing game development systems in order to assess complexity. We also offered some benchmarks that quantified certain indicators of performance and verbosity. These benchmarks compare coroutine systems across multiple scripting languages commonly used in games, in order to assess how well Casanova stacks against them in one of its most complex (and computationally intensive) sub-systems. Finally, we performed some user studies with students ranging from complete beginners in Computer Science, to advanced programmers learning the finer points of game development.

\section{Extension opportunities for Casanova}
We now discuss some of the aspects of Casanova that could be improved upon. We explain how rendering could be made more powerful, how we are building a standard library of components ready for reuse, how we are creating an IDE for supporting the Casanova language directly, and how we are extending the language in order to support general-purpose networking and AI facilities. We conclude with a note on the applicability of Casanova to AAA games.

\subsection{Rendering}
Rendering in Casanova is the weakest aspect of the current implementation and design. On one hand we believe that rendering is almost not an open problem anymore, and that much of the current research on rendering is actually focused on technical, engineering, and approximation efforts \cite{GPU_GEMS_1, GPU_GEMS_2, GPU_GEMS_3} rather than on fundamental understanding of photo-realistic lighting models \cite{CHAPTER_9_RENDERING_EQUATION}. Since we aim at solving fundamental issues with game development in general, we have chosen to create a first rendering system that allows testing of the rest of Casanova, with the objective to integrate some other rendering engine, possibly even Unity itself (or any other engine that supports .Net/Mono bindings), when the framework is sufficiently mature that the current drawing facilities become inadequate.

\subsection{Standard library}
Casanova also lacks a standard library. A standard library for a system such as Casanova would provide sets of entities that cooperate with each other in order to provide some pervasive game functionality like physics, or even whole game skeletons for different genres. Such a library would allow us to inject past experience in making certain games into Casanova, thereby further reducing the difficulty of game development with the framework; unfortunately, building such a library requires a substantial engineering effort, and is partially beyond the current scope of this work. Still, we are slowly increasing the size of the Casanova library of utilities by generalizing the various portions of the games we build when we see components that may be of broader utility.

\subsection{IDE}
The current IDE for Casanova is also a major source of future work. On one hand the implementation is lacking all the code-completion technologies that many programmers are used to, and which make coding much simpler as it allows to keep track of the source code structure automatically. Code completion also helps greatly when getting confidence with new and unknown libraries, since it makes it possible to interactively explore them without reading lots of documentation before hand. 
The lack of a proper compiler currently requires developers to write Casanova embedded in F\#. While this does not have many significant shortcomings, the language used this way loses in simplicity and power. One important disadvantage of the lack of a compiler though is that query optimizations are not active at the moment, because they require syntactic transformations that are hard to encode from F\#. Also, some small syntactic improvements (especially, but not limited to, the declaration and lookup of rules) would improve code readability. 

Further extensions that a proper language could support include a better code generator for state machines from coroutines. The current implementation instances multiple anonymous functions, namely one for each binding operation. This means that coroutines allocate memory that is often used for a very short time. Garbage collection of Casanova programs could then benefit from an optimization that uses, for example, pooling of the coroutine continuations to reduce generated garbage. Such a modification would yield a very small improvement on those implementations of the runtime that employ a modern, generational garbage collector \cite{CHAPTER_9_GENERATIONAL_GC}. The gains could be much bigger on platforms such as the XBox or tablet PCs where garbage collection uses a slower implementation \cite{CHAPTER_9_XBOX_GC_ISSUES}.

\subsection{Networking}
Networking is a major extension that is currently under construction for Casanova. A prototype implementation is  sketched in Appendix \ref{chap:networking}. Our goal is to build a system that is general, robust, and efficient. We are building such a system which, at its core, is a distributed synchronization mechanism that implements an eventually consistent serialization and deserialization of the game world from the host to the client. The eventual consistency arises from the fact that transferring the game world from the host is done many times per minute, and so information lost during one transfer will simply be expected in a future transfer. Also, the client will have to send the input events to the host in order for it to propagate the input responses to the remaining clients.

One major concern of our networking system is also performance and responsiveness. For this reason, we are building incremental transfers, data compression, and even local prediction techniques on the client side in order to minimize bandwidth and perceived delays.

\subsection{AI}
We are studying how to build a general system for AI, as described in Appendix \ref{chap:goap}. We believe that planning may be a good candidate technique for games in general, given that the simplest and most common technique for AI in games (finite state machines) is already supported with coroutines, and planning is the foundation for many reasoning systems, from path-finding to more complex decision-making. In particular, we are integrating a technique known as Goal-Oriented Action Planning \cite{APPENDIX_C_GOAP_BOOK} which allows agents to quickly find, thanks to heuristic search, action plans in order to satisfy all the preconditions that are required in order to satisfy a goal. The technique is already giving promising results in our systems, it easily incorporates path-finding, and it yields agents that seem deliberate and logical in their actions, which all contribute to reaching the final goal.

\subsection{AAA games}
An important issue with Casanova is that, for obvious constraints of time and resources, we have never developed a AAA game with it. It is understandable that AAA developers may have many reasons for not using Casanova. After all, a AAA studio has a team of programmers trained in traditional tools and languages, large libraries of existing code that should be leveraged as much as possible, and even a certain aversion to risking a big investment by using something that is not time-tested. This said, it is important to realize that Casanova helps development of games regardless of their scale. Whether they are tiny games built in a few hours, larger indie-games built in a few months, or huge AAA games that take years to develop, the same advantages of correctness, convenience, and expressive power would stand. For this reason we conclude that large development studios should at least consider the lesson of Casanova in order to take advantage of its most useful aspects within (or at least close to) the boundaries of their active development practices.


\section{Shortcomings of Casanova}
Casanova also has some shortcomings which determine what games it cannot be used to build. The main shortcomings that we have identified are: \textit{(i)} the inability to do low-level optimizations; \textit{(ii)} the required mindset shift for imperative programmers; \textit{(iii)} the rarely-seen ML syntax; and \textit{(iv)} the difficulty of expressing complex rendering tasks such as shaders.

\subsection{Low-level optimizations}
Most games, especially (but not limited to) larger titles, are sometimes faced with a need for optimization at a very low level, for example in order to run on less powerful hardware such as tablets or smartphones, or to support complex scenarios that require making use of all the available computational power.

Low-level optimizations may include control over memory allocations and deallocations, doubly-linked lists to quickly move an object from one group to another, and even re-writing some central portions of the inner loop in assembly. Unfortunately, these optimizations come at a cost: very little can be said (or controlled) about such portions of code, which would possibly break all the other semantic structures of the language. We argue that supporting low-level optimizations should be done only if we can at the same time retain all the properties of the system. Alternatively, we believe that better automated optimizations could further remove the need or desire to perform such optimizations. In the end though, Casanova is explicitly not aimed at AAA games, which are the only ones employing such aggressive optimizations techniques; for this reason Casanova emphasizes high-level abstraction at the expense of low-level optimization.

\subsection{Imperative mindset shift}
Casanova is a hybrid language that uses declarative/functional primitives for a large part of the game (namely, rules). The fact that rules may not affect any entity excluded the one they are operating on may cause some difficulties in programmers used to imperative operations that allow them to potentially modify any entity in the program from virtually any place in the code. This form of programming though may encounter difficulties in scaling with larger projects, since complex side-effects may get out of hand and break invariants that are otherwise assumed. Learning a declarative/functional way of thinking about programs can be a valuable skill in that it allows to cleanly split functionality in such a way that testing may be easier, and also that reasoning on the program may be easier as well. This said, programmers coming from an imperative background may require a larger initial effort than people with no background in programming at all, as we have discussed in Chapter \ref{chap:evaluation}.

\subsection{Unusual syntax}
Similarly to the issues with the declarative/functional style of programming, the syntax of Casanova is based on the lesser known syntax of ML and F\#. This syntax is widely used nowadays, but not as much as that of other imperative languages such as C, Java, C\#, or Python.
Building an alternate syntactic front-end would not change the underlying semantics or possibly even the backend implementation of our system, but it would still require substantial work in both design and implementation. Using a more common syntax, though, may reduce the perceived steepness of the learning curve of Casanova and thus remains a desirable feature.

\subsection{Advanced rendering}
Programming complex rendering operations is, as of now, delegated to the underlying rendering mechanisms of the graphics engine used by Casanova. We argue that it would be desirable to be able to express complex rendering operations in Casanova itself, for example by defining shaders, render target, etc. with separate syntactic abstractions. Unfortunately this would require a major design and implementation effort, and as such it must be relegated to our future intentions.

\subsection{Other languages}
As our implementation language, we picked F\#. The choice of F\# was motivated by its balance of performance, game development libraries, IDE support, and meta-programming through monads and reflection. Even though our experience in using our own libraries confirms F\# to be a good choice, there are multiple aspects that could not be translated from the Casanova language. 

Powerful languages such as C++ and Haskell are a good fit for a game development system such as Casanova. Interestingly, neither of them is capable of expressing Casanova constructs easily, and both require significant effort. Haskell would need an extension to the language (namely Generic Haskell \cite{APPENDIX_E_GENERIC_HASKELL}), while the degree of meta-programming used in C++ creates problems with the type inference of template parameters and often results in incomprehensible error messages. 

Haskell type-classes, in particular, allow us to code advanced meta-programming constructs, while still retaining some degree of assistance from the compiler in terms of meaningful error messages, partial compilation (to speed up compile-times for large programs), and other similar advantages. Of course, modern pragmatic languages such as C\#, F\#, Java, and the like allow the exploration of types at run-time with dynamic, unsafe operations known as \textit{reflection}. This allows these languages to retain high levels of expressivity for complex patterns (such as dependency injection), but with the penalty of lower run-times and no validation from the compiler (programs that use reflection may encounter unforeseen and catastrophic failures if used naïvely). Thanks to reflection and monads, Casanova is currently implemented as an embedding into F\#.

We make one final remark about the choice of F\# instead of a dynamic language that would not have required the "contorsions" of reflection to implement rules. For example, in Python, invoking rules would have been much simpler and would not have used reflection; moreover, Python (partially) supports coroutines and suspension mechanisms with its feature of generators.

\textit{These difficulties in adapting existing languages to Casanova suggests that Casanova is indeed a novel solution that offers significant features that are orthogonal to those present in existing languages.}

We leave as a challenge the successful implementation of Casanova inside other existing languages.
