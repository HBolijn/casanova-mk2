%%%%%%%%%%%%%%%%%%%%%%%%%%%%%%%%%%%%%%%%%%%%%%%%%%%%%%%%%%
% case_study.tex
%%%%%%%%%%%%%%%%%%%%%%%%%%%%%%%%%%%%%%%%%%%%%%%%%%%%%%%%%%

%%%%%%%%%%%%%%%%%%%%%%%%%%%%%%%%
%edit
%%%%%%%%%%%%%%%%%%%%%%%%%%%%%%%%
In this section we will describe how we have rewritten the XNA Spacewar \cite{XNA_SAMPLES} sample in Casanova, the resulting reduction in code and the increases in performance obtained. We have chosen Spacewar because it is small enough to be didactically useful while being built as a starter kit, that is a starting point to be edited and extended into a different game and not just as a sample or tutorial; from this point of view Spacewar should be considered as a small, yet complete and well-built, game.

The Casanova compiler is still in its very early stages, and as such it is not yet ready for the task. The definition of the compiler can be followed by hand, and since the first Casanova compiler will generate F\# code, we have written such code by hand as the compiler would have output it.

\subsection{Rewriting the Game}
The original sample features two ships that shoot each other while dodging asteroids that float around the gaming area. A star in the center of the playing field pulls the players with its gravity. The first player to destroy the other (by hitting him or by making him crash on another celestial body) wins the stage.

The game state is defined as the two players, their ships, the table of asteroids and projectiles and the sun. Also, the state contains the current gameplay status, which can either be \texttt{Playing} or \texttt{GameOver w} where \texttt{w} is the winner.

%%%%%%%%%%%%%%%%%%%%%%%%%%%%%%%%
%edit
%%%%%%%%%%%%%%%%%%%%%%%%%%%%%%%%
The source code of the original sample plus our implementation can be found in \cite{CASANOVA_CODEPLEX}; the current implementation of the Casanova compiler is incomplete, and at the time of writing the type checker and the F\# code generator are both producing their first correct outputs but are not yet integrated together. The details of the porting are discussed in detail in \cite{CASANOVA_TR}, and we omit them here for reasons of space.

We have slightly modified the original sample so that testing could be automated. For this reason we have removed the 30 seconds time limit of each level, we have removed the victory and ending conditions, we have automated ships movement and shooting and we have increased the maximum number of asteroids and projectiles to 12 and 200 respectively. This way we have obtained an automated stress test.

%%%%%%%%%%%%%%%%%%%%%%%%%%%%%%%%
%edit
%%%%%%%%%%%%%%%%%%%%%%%%%%%%%%%%
We have also removed all rendering features, to avoid benchmarking rendering algorithms: Casanova does not generate rendering code, so such a comparison would have been meaningless; also, Casanova can be integrated with the very same C\# rendering code of the original Spacewar. We compare the resulting framerates to see how many simulation steps per second the original game logic is capable of performing versus the number of steps per second of the Casanova game logic; the higher this number, the more efficient the game logic and the more time remains for each frame to perform complex rendering.

As a final remark, it is worth noticing that while the original sample includes more than one thousand lines of code the length of the corresponding Casanova program is 348 lines long. The Casanova source easily fits a few pages, while navigating the original source may prove a bit complex because of its sheer size.

\subsection{Resulting Benchmarks}

We have benchmarked the modified sample on both the Xbox 360 and a 1.86 Ghz Intel Core 2 Duo with an nVidia GeForce 320M GPU and 4GB of RAM. In the table below we can see the framerates of the various tests.

\begin{table}[ht] 
\center
\begin{tabular}{|c|c|c|c|c|c|} 
\hline
C\# XBox & C\# PC & Casanova XBox & Casanova PC \\ 
\hline
8   & 9   & 22  & 577 \\
\hline
\end{tabular} 
\caption{Framerate of the original Spacewar vs the Casanova implementation}
\end{table}

As we can see, full Casanova optimization always beats the original source by at least a factor of 2. The Xbox implementation suffers from the generation of garbage, which is a known problem of the XNA implementation on the console (\cite{XBOX_GC}); indeed, profiling the garbage collector shows that large amounts of temporary memory are being generated by the program. It is noticeable that on the PC, thanks to the full optimizations done by Casanova, performance increased by almost two orders of magnitude: such an impressive increase was quite unexpected even by us, more so when keeping in mind that those optimizations will be automated by the compiler.
