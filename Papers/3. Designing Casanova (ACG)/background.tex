%%%%%%%%%%%%%%%%%%%%%%%%%%%%%%%%%%%%%%%%%%%%%%%%%%%%%%%%%%
% background.tex
%%%%%%%%%%%%%%%%%%%%%%%%%%%%%%%%%%%%%%%%%%%%%%%%%%%%%%%%%%

In this section we discuss some current approaches to game development.

The two most common game engine architectures found in today's commercial games are object-oriented hierarchies and component-based systems.

In a traditional object-oriented game engine the hierarchy represents the various game objects, all derived from the general \texttt{Entity} class. Each entity is responsible for updating itself at each tick of the game engine \cite{OO_GAME_DEV}.

A component-based system defines each game entity as a composition of components that provide reusable, specific functionality such as animation, movement, reaction to physics, etc. Component-based systems are being widely adopted, and they are described in \cite{COMPONENTS1}.

These two, more traditional approaches both suffer from a noticeable shortcoming: they focus exclusively on representing single entities and their update operations in a dynamic, even composable way. By doing so they lose the focus on the fact that most entities in a game need to \textit{interact} with one another (collision detection, AI, etc.), and usually lots of a game complexity comes from defining (and optimizing) these interactions. Also, all games feature behaviors that take longer than a single tick; these behaviors are hard to express inside the various entities, which often end up storing explicit program counters to resume the current behavior at each tick.

There are two more approaches that have emerged in the last few years as possible alternatives to object-orientation: (functional) reactive programming and SQL-style declarative programming.

Functional reactive programming (FRP, see \cite{FRP}) is often studied in the context of functional languages. FRP programming is a data-flow approach where value modification is automatically propagated along a dependency graph that represents the computation. FRP offers a solution to the problem of representing long-running behaviors, even though it does not address the problem of many entities that interact with each other.

SQL-queries for games have been used with a certain success in the SGL language (see \cite{SCALING_GAMES}). This approach uses a lightweight, statically compiled query engine for defining a game. This query engine is aggressively optimized, in order to make it simple to express efficient aggregations and cartesian products, two very common operators in games. On the other hand, SGL suffers when it comes to representing long-running behaviors, since it focuses exclusively on defining the tick function.

We have designed Casanova with these two issues in mind: with Casanova, the integration of the interactions between entities and long-running behaviors is seamless.