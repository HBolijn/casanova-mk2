In this paper we propose the Resource Entity Action (REA) pattern to define RTS games. Use of this pattern should protect developers from writing and rewriting large amounts of boilerplate code. The paper presents:

\begin{itemize}
\item the \textbf{REA design pattern} for making RTS games which reduces the interaction among entities to a dynamic exchange of resources;
\item an expressive, declarative, high performance \textbf{language extension} to Casanova, with an appropriate grammar with new syntax and semantics resembling SQL;
\item an evaluation with three examples which provides evidence for an increase in programming efficiency using REA; and
\item an evaluation that shows an increase in run time efficiency of 6 to 25 times for the Casanova language, using a native code compiler/opimizer.
\end{itemize}

In future work, we estimate that even better results can be obtained with an actual access plan optimizer that increases the performance when exploring the structure of both the action query and the entity structure. Given the significant results on position indexing, the chance of defining multi-attribute indexes would increase the performance.


%This paper presents the following contributions. Our technique is shown to reduce the amount of boilerplate code that needs to be implemented, tested, and debugged, therefore yielding increased development productivity. In this paper we provide evidence that our approach is (\textit{i}) syntactically simpler, (\textit{ii}) semantically powerful, (\textit{iii}) general purpose, (\textit{iv}) high-level, and (\textit{v}) high-performance. 