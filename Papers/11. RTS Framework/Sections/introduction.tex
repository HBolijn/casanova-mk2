Real-time strategy (RTS) games have been highly popular for decades. As outlined by the ESA \cite{ESAreport}, RTS games are registering strong sales and a large number of play hours. Commercial RTS games are written by game developers of different backgrounds: from large studios to smaller independent developers of \textit{indie games}. Indie developers \citep{IndiGame} typically consist of small teams and their games are known for innovation \citep{INDIEinnovation}, creativity \citep{INDIEcreativity} and artistic experimentation \citep{INDIEexperimentation}. RTS games are also built as ``serious games'' \citep{SeriousRTS}, used for training and education, and as ``research games'' \citep{ResearchRTS}.

In general, the building of games is an expensive venture \citep{CostsRTS}. This is challenging in particular for indie developers and developers of serious and research games, who usually have access to few resources. They would benefit of cost-effective development methodologies for games, through the identification and automation/reuse of common patterns in games. Surprisingly, from a survey of game development research and literature, we noticed a lack of studies of abstract patterns which characterize games, in particular RTS games. This motivates our research question: \textit{to what extent can we capture the commonalities of RTS games in a re-usable design pattern?}

Section 2 discusses the essential elements of an RTS game. Section 3 specifies the \emph{Resource Entity Action} (REA) design pattern  \cite{designpatterns} that captures these essential elements. Section 4 describes how the pattern is implemented as a language extension of the Casanova game programming language \citep{Casanova}. The language extension is purely declarative, using semantics that resemble SQL, providing an intuitive adoption for most programmers. We implemented the extension in a full-fledged RTS, which we discuss and analyze in Section 5. %In Section 6 we make an analytic comparison between several other available solutions for making RTS games and our own.

%This paper presents the following contributions. Our technique is shown to reduce the amount of boilerplate code that needs to be implemented, tested, and debugged, therefore yielding increased development productivity. In this paper we provide evidence that our approach is (\textit{i}) syntactically simpler, (\textit{ii}) semantically powerful, (\textit{iii}) general purpose, (\textit{iv}) high-level, and (\textit{v}) high-performance. 