RTS are a variation of strategy games where two or more players achieve specific (often conflicting) objectives by performing actions simultaneously in real time. The typical elements which arise from this genre are \textit{units} (characters, armies), \textit{buildings}, \textit{resources} and \textit{battle statistics}. Players command units to perform different types of actions. These actions can affect several entities in the game world.

Units and buildings are the entities that players control to achieve their objectives. Units usually fight or harvest resources, while buildings may be used to create new units or research upgrades. Resources are gathered from the playing field and fuel the economy of the game entities. Battle statistics determine the offensive and defensive abilities of units in a fight. This taxonomy of the elements of an RTS game can be applied successfully to multiple games: Starcraft, C\&C, and Age of Empires all feature units, buildings, resources, and battle statistics, amongst other elements.

In order to arrive at our design pattern we will now apply a simplification. Battle statistics can be interpreted as \textit{resources}, as for instance: ``the life of a unit is the cost for killing it, payable in attack power.'' We can also merge units and buildings together into a new category called \textit{entities}. This leads us to a simpler view of an RTS as a game that is based on Resources, Entities and Actions:

\begin{enumerate}
\item Resources: numerical values in the battle and economic system of the game. In this group we find the \textit{attack}, \textit{defense}, and \textit{life} patterns of entities. Resources also cover building materials and costs of production, deployment of units, development of new weapons, etc. (Resources are scalars.)
\item Entities: container for resources. They have physical properties and, as for the game logic, the difference among them lies only in the interactions. These interactions take place with resource exchanges through the actions. (Entities are vectors.)
\item Actions: resource flow among entities. Our model can be viewed as a directed weighted graph where the nodes are the entities, the weights are the amounts of exchanged resources, and the edges are the actions, that is, the elements which connect entities to one another. (Actions are transformation matrices.)
\end{enumerate}

Next, we discuss how we model Resources, Entities and Actions. 