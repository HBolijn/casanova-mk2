%% Changed by PS, April 4, 2014.

\section{Future work}
\label{sec:future_work}
The Casanova 2 language is capable of implementing usable and quite complex games. The language, while usable, is currently still in development as it misses a few features. In particular, support for multiplayer games is at this moment lacking. We believe that the existing mechanisms for handling time offered by Casanova 2 could be augmented with relatively little effort in order to greatly simplify the hard task of building multiplayer games. This is part of future work, that we are currently engaging in. We are also doing usability studies using students from various disciplines and backgrounds.

The high level view of the game that the Casanova 2 compiler provides can be exploited in order to improve the programmer experience. This means that we could use tools for code analysis (such as abstract interpretation \cite{nielson1999principles} or type system extensions) in order to better understand the game being built, and to help with correctness analysis, performance analysis, or even optimization.


%\subsection{User study}
%We wish to perform an in-depth user study for Casanova 2 to improve usability in the development process. We have already performed a partial (and quite promising) small user study which we will extend and complete.


%We have performed the following test: we gathered a group of students of game programming and a group of students of game design. We gave them a series of Casanova 2 samples, printed on paper. Each student had to guess the functionality of each sample, and sketch a screen-shot. Furthermore, each student also provided some additional feedback on the language.

%The samples were: (\textit{i}) a string of text moved around the screen with the keyboard, (\textit{ii}) a string of text that moves along a predefined path automatically, and (\textit{iii}) an asteroid shooter.

%Eleven (over a total of thirteen) students understood the samples completely, both drawing the screen-shots and explaining the dynamics of the game correctly. Two students were lost on the syntactic differences between Casanova 2 and the more familiar C-like syntax. The direct feedback was mostly centred around a series of common observations, which are reported in Table \ref{students_feedback}. For each observation, the table reports how many times we encountered it.

%\begin{table}[!t]
%% increase table row spacing, adjust to taste
%\renewcommand{\arraystretch}{1.3}
% if using array.sty, it might be a good idea to tweak the value of
% \extrarowheight as needed to properly center the text within the cells

%\caption{Feedback from students}
%\label{students_feedback}
%\centering

%% Some packages, such as MDW tools, offer better commands for making tables
%% than the plain LaTeX2e tabular which is used here.
%\begin{tabular}{|c||c|}
%\hline
%Syntax is unfamiliar at first & 3\\
%\hline
%Syntax is clear & 8\\
%\hline
%Indentation instead of parentheses is a downside & 2\\
%\hline
%List processing with queries is very effective & 1\\
%\hline
%Rules are a good abstraction for games & 2\\
%\hline
%\end{tabular}
%\end{table}

%We also built a significantly bigger sample, which we asked only three students to study. The sample is a checkpoint-based RTS (see Figure \ref{RTS game} for a screenshot). All students correctly identified the game mechanics, and provided some additional feedback. Most of this feedback overlaps with that obtained for the samples, but some new observations emerge. Arguably, some patterns become visible only with larger samples:
%\begin{itemize}
%\item \texttt{wait} and \texttt{when} are very powerful
%\item Multiple rules on the same field are very powerful
%\item Multiple rules on the same field may lead to behaviours that are complex to understand
%\end{itemize}


\section{Conclusions}
\label{sec:conclusions}

Casanova 2, a language specifically designed for building computer games, may offer a solution for the high development costs of games. The goal of Casanova 2 is to reduce the effort and complexities associated with building games. Casanova 2 manages the game world through entities and rules, and offers constructs (wait and yield) to deal with the run-time dynamics. As shown by the benchmarks in Section \ref{sec:evaluation}, we believe that we have taken a significant step towards reaching these goals. In fact, we achieved at the same time very good performance and simplicity, thereby empowering developers with limited resources. 