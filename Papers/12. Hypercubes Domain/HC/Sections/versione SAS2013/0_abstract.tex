Application domains like Computer Games Software are an interesting workbench to stress the trade-off between accuracy and efficiency of abstract domains for static analysis, in the Abstract Interpretation framework. Game software deeply relies on physics simulations, which are particularly demanding: due to the large amount of interleaving floating point variables, the numerical domains already studied in the literature either lack accuracy too much, or their use becomes unfeasible due to the exponential cost of abstract operations.

In this paper, we discuss the domain of Parametric Hypercubes, a novel disjunctive non-relational abstract domain. Its main features are: (i) it combines the low computational cost of operations on (selected) multidimensional intervals with the accuracy provided by lifting to a power-set disjunctive domain, (ii) the compact representation of its elements allows to limit the space complexity of the analysis, and (iii) the parametric nature of the domain provides a way to tune the accuracy/efficiency of the analysis by just setting the widths of the hypercubes sides.

The first experimental results on a representative Computer Games case study outline both the efficiency and the precision of the proposal.

