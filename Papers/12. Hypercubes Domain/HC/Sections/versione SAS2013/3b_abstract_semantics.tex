For the most part, the abstract semantics applies existing semantic operators of boxed Intervals \cite{COU98}.

First of all, $\mathbb{I}$ defines the semantics of arithmetic expressions on a single hypercube by applying the well-known arithmetic operators on intervals.

This semantics is used to define the abstract semantics $\mathbb{B}$ of Boolean comparisons. Given a hypercube and a Boolean comparison $E_1 <bop> E_2$ where $<bop> \in \{\geq, >, \leq, <, \neq \}$, this semantics returns an \emph{abstract value of the boolean domain} (namely, $\mathit{true}$, $\mathit{false}$, or $\top$) comparing the intervals obtained from $E_1$ and $E_2$ through $\mathbb{I}$. Therefore, given a Boolean condition and a set of hypercubes, we partition this set into the hypercubes for which (i) the condition surely holds, (ii) the conditions surely does not hold, and (iii) the condition may or may not hold. In this way, we can discard all the hypercubes for which a given Boolean condition surely holds or does not hold.

Note that in this way we lose some precision. For instance, imagine that in a given hypercube we know that $\statement{x} \in [0..5]$ (because the fixed width of intervals associated to \statement{x} is 5), and we check if $\statement{x} \leq 3$. The answer of $\mathbb{B}$ will be $\top$ and, if the condition was inside an \statement{if} statement, this hypercube will be used to compute the semantics of both the branches. Indeed, we would know that $\statement{x} \in [0..3]$ in the \statement{then} branch, and $\statement{x} \in [4..5]$ in the else branch. Nevertheless, we cannot and do not want to track this information in our hypercube, since the width of the interval associated to \statement{x} is 5, and fixed widths are a key feature in order to obtain an efficient analysis. We will present (Section \ref{sec:widths}) how we can recursively modify the widths of the analysis to improve precision in these cases.

The semantics of the logical operators $not$, $and$, $or$ is defined in the standard way.

$\mathbb{I}$ is used to define the semantics $\mathbb{S}$ of variable assignment as well. The standard semantics of \statement{x=exp} is to (i) obtain the interval representing the right part ($\mathbb{I}\llbracket\statement{x=exp}, \sigma\rrbracket=[m..M]$), and (ii) assign it in the current state. This approach does not necessarily produce a single hypercube, since we could assign an interval which width is greater than the fixed width of the assigned variable, or that covers several other intervals belonging to different partitions. In this case, we build up several hypercubes that cover the interval $[m..M]$. This can be formalized by $assign(h, V_i, [a..b])=\{h[i\mapsto m] : [m\times w_i .. (m+1)\times w_i]\cap [a..b] \neq \emptyset\}$, where $h$ is a hypercube, $V_i$ is the assigned variable, and $[a..b]$ is the interval we are assigning (which depends on the hypercube $h$, since we use its variables values to compute the result of the expression). We repeat this process for each hypercube $h$ in the abstract state by using it as input for the computation of $assign$.
In this way, we are able to over-approximate the assignment keeping the fixed widths of the intervals, which is a key feature of our domain in order to obtain an efficient analysis.

%\todogiulia{Tino: come dicevo questa sezione si dovrebbe ridurre all'osso: riferimenti a Intervalli, Disjuctinve domains trough Powerset, Prodotto Cartesiano, Conditional Partitioning}
%
%In this Section, we define the abstract semantics of Hypercubes. In particular, the semantics we will define are the following ones:
%\begin{itemize}
%\item $\mathbb{I}$, the abstract semantics of arithmetic expressions, which receives an expression and a single hypercube in input and returns an \emph{interval of real values} resulting from the execution of that expression when the variable values belong to that hypercube.
%\item $\mathbb{B}$, the abstract semantics of boolean conditions, which receives a single hypercube and two expressions in input and returns an \emph{abstract value of the boolean domain} (namely, $true$, $false$, or $\top$) obtained by comparing the two expressions (through $\geq, >, \leq, <$ or $\neq$) when the variable values belong to that hypercube. Boolean conditions can be combined through logical operators (and, or, not) in the usual way (i.e., exploiting the abstract semantics of the boolean abstract domain). 
%\item $\mathbb{S}$, the abstract semantics of statements, which receives in input a set of hypercubes (the current abstract state) and returns a new set of hypercubes (the new abstract state after the execution of the statement).
%\end{itemize}
%
%\subsection{The abstract semantics of arithmetic expressions, $\mathbb{I}$}
%
%\subsubsection{Constants}
%\label{sec:constants}
%We define a constant as a variable which gets assigned only once with a constant value, or a numerical value which appears in some statements (without being assigned to a specific variable). To simplify the treatment of constants, we execute a preprocessing on the program with constant propagation, to remove constant variables and replace their uses with their numerical value. 
%
%
%The abstract semantics of an expression made up by a constant numeric value is, simply, an interval of zero width: the extremes of the interval are the same and they are equal to the value of the constant. 
%Then, the abstract semantics of a constant is:
%
%$$\mathbb{I}\llbracket c \rrbracket \; h = [c,c]$$
%
%Note that the value of the hypercube in input ($h$) is ignored because it is not needed to compute the result.
%
%\subsubsection{Intervals}
%The abstract semantic of an expression made up by an interval of real values is immediate: it is exactly that interval, without modifications. We ignore the hypercube passed in input.
%
%$$\mathbb{I}\llbracket (m,M) \rrbracket \; h = [m,M]$$
%
%\subsubsection{Variables}
%When the expression is made up by a variable, we must consider the abstract value of that variable in the hypercube passed in input. Let $h_i$ be the integer index associated to the $i$-th dimension of the hypercube, $V_i$ the $i$-th variable defined in the program and $w_i$ the width associated in the analysis to such variable. Then, the abstract semantics of a variable  is $\mathbb{I}\llbracket V_i\rrbracket \; h = [h_i*w_i,(h_i+1)*w_i]$.
%
%
%\subsubsection{Arithmetic operations}
%We considered only the most used arithmetic operators. In particular, we considered sum ($+$), subtraction ($-$), product ($\times$) and division ($\div$). These operators should suffice for most physics simulations (for example, our case study requires only sum and product - the change of sign being a multiplication for $-1$). Anyway, our framework can be easily extended to support other operations (for example modulus), by simply defining their abstract semantics when working on intervals of values.
%
%The semantics of these operators over intervals has been already deeply studied, and we refer to \cite{COU98} for their definitions.
%
%
%\subsection{The abstract semantics of Boolean conditions, $\mathbb{B}$}
%\label{sec:boolcondition}
%We now define the semantics of integer comparison using the operators $>$, $<$, $\leq$, $\geq$, and $\neq$. 
%
%The abstract semantics separately computes the semantics on the left and right arithmetical expressions of the statement, obtaining two intervals. Let us call them $i_1 = [a,b]$ and $i_2 = [c,d]$. The abstract comparison between them depends on the specific comparison operator present in the statement:
%\begin{itemize}
%\item $i_1 \neq i_2$ returns true if $b < c \vee a > d$, false if $b=c=a=d$, and $\top$ otherwise.
%\item $i_1 < i_2$ returns true if $b < c$, false if $a > d$, top otherwise.
%\item $i_1 > i_2$ returns true if $a > d$, false if $b < c$, top otherwise.
%\item $i_1 \leq i_2$ returns true if $b \leq c$, false if $a \geq d$, top otherwise.
%\item $i_1 \geq i_2$ returns true if $a \geq d$, false if $b \leq c$, top otherwise.
%\end{itemize}
%
%$\mathbb{B}$ will be used by the statement semantics $\mathbb{S}$ when computing the semantics of \statement{if} and \statement{while} statements to discard the hypercubes that surely do not satisfy the condition. Note that in this way we lose some precision. For instance, imagine that in a given hypercube we know that $\statement{x} \in [0..5]$ (because the fixed width of intervals associated to \statement{x} is 5), and we check if $\statement{x} \leq 3$ when computing the semantics of an \statement{if} statements. The answer of $\mathbb{B}$ will be $\top$, and so this hypercube will be used to compute the semantics of both the branches. Indeed, we would know that $\statement{x} \in [0..3]$ in the \statement{then} branch, and $\statement{x} \in [4..5]$ in the else branch. Nevertheless, we cannot track this information in our hypercube, since the width of the interval associated to \statement{x} is 5. Anyway, we will present (Section \ref{sec:widths}) how we can recursively modify the widths of the analysis to improve precision in these cases.
%
%
%\subsection{The abstract semantics of statements, $\mathbb{S}$}
%
%\subsubsection{Assignment}
%Usually, with non-disjunctive domains, the abstract semantics of an assignment is straightforward: you have to compute the abstract semantics of the expression and update the abstract value of the variable with the result. In our domain, though, we track a different kind of information: we represent possible values of all variables together (through hypercubes) and we consider disjunctive information (a set of valid hypercubes instead of a single hypercube). Therefore, we must devise a specific abstract semantics to deal with the assignment statement. 
%
%Let the assignment be $V_i = e$, where $e$ is an arithmetic expression. Our approach can then be sketched as follows: 
%\begin{itemize}
%\item we consider, separately, each hypercube $h$ of the current state;
%\item we compute the abstract semantics of the arithmetic expression $e$ passing to it the hypercube $h$.
%\item we create a new hypercube (or \emph{some} new hypercubes, depending on the width of the resulting interval), where the abstract value of $V_i$ is the abstraction of the interval resulting from $e$.
%\end{itemize}
%
%If the interval obtained from $e$ overlaps over several partitions established for variable $V_i$, we have to build up several hypercubes as result of the assignments semantics over a single hypercube. 
%
%
%On the other hand, if many hypercubes of the initial state map to the same hypercube in the resulting state, it could also happen that the cardinality of \adomain\ decreases (or remains unmodified) after the execution of an assignment.
%
%Formally: 
%
%$$\mathbb{S}\llbracket \statement{V_i = e}\rrbracket \; H = \bigcup_{h \in H} {assign(h, V_i, \mathbb{I}\llbracket e\rrbracket \; h)}$$
%
%where $assign(h, V_i, [a,b])=\{h[i\mapsto m] : [m\times w_i .. (m+1)\times w_i]\cap [a..b] \neq \emptyset\}$. The output of this function is the set of hypercubes covering all the intervals that overlaps with the interval assigned to the given variable.
%
%
%\subsubsection{If-then-else semantics}
%
%To precisely deal with branches of \statement{if} statements, we partition the abstract state \adomain\ with respect to the evaluation of the branching condition. In particular, we compute the abstract semantics $\mathbb{B}$ of the boolean condition on each hypercube of \adomain\ and we assign each hypercube to a specific partition, based on the result of the condition semantics. Therefore, we obtain three partitions: the hypercubes for which the condition evaluates to true ($p_t$), the ones for which the condition evaluates to false ($p_f$), and the ones for which we do not have a definitive answer ($p_\top$).
%
%Once obtained these three partitions, we can compute selectively the abstract semantics of the two branches, and in particular the \statement{then} branch with $p_t \cup p_\top$, and the \statement{else} branch with $p_f \cup p_\top$. 
%
%Formally, the semantics of the \statement{if} statements follows:
%
%$$\mathbb{S}\llbracket \statement{if}(B)\ \statement{then}\ P_1\ \statement{else}\ P_2\rrbracket \; H = ( \mathbb{S}\llbracket P_1\rrbracket \; (p_t \cup p_\top) ) \cup ( \mathbb{S}\llbracket P_2\rrbracket \; (p_f \cup p_\top) )$$
%
%where $p_t = \{ h \in H : \mathbb{B}\llbracket \ael{if}\rrbracket \; h = true \}$, $p_f = \{ h \in H : \mathbb{B}\llbracket \ael{if}\rrbracket \; h = false \}$, and $p_\top = \{ h \in H : \mathbb{B}\llbracket \ael{if}\rrbracket \; h = \top \}$.
%
%\subsubsection{Concatenation of statements}
%The abstract semantics of the concatenation of two statements is straightforward: it executes the abstract semantics of the first statement, it takes the result and it passes it as input to the abstract semantics of the second statement. Formally:
%
%$$\mathbb{S}\llbracket P_1;P_2\rrbracket \; H = \mathbb{S}\llbracket P_2\rrbracket \; (\mathbb{S}\llbracket P_1\rrbracket \; H)$$
%
%\todogiulia{Non sono per niente convinto che sia questa big-step semantics quello di cui abbiamo bisogno. Ad esempio, la semantics di \statement{while(true) P_1; P_2} ritorna $\emptyset$ stando a questa definizione. A occhio abbiamo bisogno di qualcos'altro, e.g., uno stato astratto per ogni punto di programma di un cfg? In questo caso, non dobbiamo lavorare a questo livello con \statement{if} e \statement{while} ma piu' a basso livello.}
%
%\subsection{While loop}
%To do. 

