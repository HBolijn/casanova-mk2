In this paper we presented Parametric Hypercubes, a disjunctive non-relational abstract domain which can be used to analyse physics simulations. Experimental results on a representative case study show the precision of the approach. The performance of the analysis makes it feasible to apply it in practical settings. 

Note that our approach offers plenty of venues in order to improve its results, thanks to its flexible and parametric nature. In particular, we could: (i) increase the precision by intersecting our hypercubes with arbitrary bounding volumes which restrict the relationships between variables in a more complex way than the offsets presented in Section \ref{sec:semantics}; (ii) increase the performance of Algorithm \ref{alg:widthAdjusting} by halving the widths only on some axes, chosen through an analysis of the distribution of hypercubes in the \emph{yes,no,maybe} sets; and (iii) study the derivative with respect to time of the iterations of the main loop in order to define temporal trends to refine the widening operator.
In addition, our domain is modular w.r.t. the non-relational abstract domain adopted to represent the hypercube dimensions. By using other abstract domains it is possible to track relationships between variables which do not necessarily represent physical quantities.