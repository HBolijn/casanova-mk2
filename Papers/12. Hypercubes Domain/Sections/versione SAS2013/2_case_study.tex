As a case study, consider a program which generates a bouncing ball on the screen. The ball starts at the left side of the screen (even though the exact starting position is not fixed), and it has a random initial velocity. The horizontal direction of the ball is always towards the right of the screen, since the horizontal component of the velocity is always positive. Whenever the ball reaches the \textquotedblleft ground\textquotedblright\ (i.e., the bottom of the screen), it bounces (i.e., its vertical velocity is inverted). When the ball reaches the right border of the screen, it disappears. We want to verify that $T$ seconds after the generation of the ball, such ball has already exited from the screen (we call this property \emph{Property 1}). The code of the program which simulates the creation and movement of the ball is reported in Listing \ref{lst:casestudy}. 

\begin{lstlisting}[caption={Case study: bouncing-ball code},label={lst:casestudy}]
let px = rand(0.0, 10.0)
let py = rand(0.0, 50.0)
let vx = rand(0.0, 60.0)
let vy = rand(-30.0, -25.0)
let dt = 0.05
let g = -9.8
let k = 0.8

while (true) do
   if( py >= 0.0 ) then 
     (px, py) = (px + vx * dt, py + vy * dt)
     (vx, vy) = (vx, vy + g * dt)
   else
     (px, py) = (px + vx * dt, 0.0)
     (vx, vy) = (vx, -vy) * k
\end{lstlisting}

The structure of this program respects the generic structure of a physics simulation, as explained in Section \ref{sec:introduction}. In fact, it starts with the world initialization, that is, the assignment of the initial values to the program variables. The meanings of the variables are as follows:
\begin{itemize}
\item \statement{(px,py)} represents the current position of the ball in the screen. The initial position of the ball is generated randomly. We only know that the ball is generated in proximity of the left side of the screen (since the $x$-coordinate is between $0.0$ and a low value such as $10.0$).
\item \statement{(vx,vy)} represents the current velocity of the ball. The initial velocity is generated randomly as well. Note that the horizontal velocity is always positive, because we want to throw the ball towards the right of the screen. The vertical velocity, instead, is negative, because we throw the ball downwards.
\item \statement{dt} represents the time interval between iterations of the loop. This value is constant and known at compile time. We consider a simulation running at 20 frames per second: this means that $\statement{dt}=1/20=0.05$.
\item \statement{g} represents the force of gravity ($-9.8$).
\item \statement{k} represents how much the impact with the ground decreases the velocity of the ball.
\end{itemize}

The \statement{while} loop updates the world variables, that is, the ball position and velocity. In particular, if the ball is in the air (i.e., its vertical position is greater or equal to $0.0$), its position is updated according to the rule of uniform linear motion. The vertical velocity is also updated to take into account the force of gravity, while the horizontal one remains unmodified. Otherwise, the ball touches the ground\footnote{We must wait that the position of the ball is \emph{lower} than zero before making it bounce, otherwise a ball could bounce before having actually touched the ground. The ball could disappear from the screen for a bit, but, considering the frame-rate of modern games, this artifact is almost not perceived by the user.} and it bounces. To simulate the bouncing, we update the horizontal position as usual, and we force the vertical position to zero. Moreover, the vertical velocity is inverted (instead of going downwards, the ball must go upwards) and decreased, along with the horizontal velocity, through the constant factor \statement{k}, to consider the force which is lost in the impact with the ground. 

Verifying properties on this kind of programs can have interesting practical impact also on more complex programs, since it is a basic physics simulation which can be used in many contexts \cite{E10}. For instance, consider the case where some game entity discreetly generates bouncing balls on the screen. The interval between the creation of two balls is constant and known at compile time (let us call it \statement{creationInterval}). The pseudo code of the main method of this simulation is shown in Listing \ref{lst:casestudyExternal}, where \statement{updateBall(b)} is a function which updates the ball \statement{b} according to the physics of the simulation (i.e., the body of the while loop of Listing \ref{lst:casestudy}) and \statement{generateNewBall()} is a function which creates a new ball (with the values of the initialization of Listing \ref{lst:casestudy}).

\begin{lstlisting}[caption={Bouncing ball generation},label={lst:casestudyExternal}]
let balls = Set.empty
let dt = 0.05
let creationInterval = 3.0
let timeFromLastCreation = 0.0
while (true) do
	foreach ball in balls
		updateBall(ball)	
	if(timeFromLastCreation >= creationInterval)
		generateNewBall()
		timeFromLastCreation = 0.0
	else
		timeFromLastCreation += dt
\end{lstlisting}

If we verify \emph{Property 1} on Listing \ref{lst:casestudy}, we are sure that a single ball will have exited the screen after $T$ seconds. We also know that, in Listing \ref{lst:casestudyExternal}, we generate one ball each $\statement{creationInterval}$ seconds. This means that, having verified \emph{Property 1}, we can guarantee in Listing \ref{lst:casestudyExternal} that \emph{a maximum of $\lceil \frac{T}{\statement{creationInterval}} \rceil$ balls will be on the screen at the same time}. Such information is useful for performance reasons (crucial in a game), since each ball requires computations for its rendering and updating. 

%Thus, what we would like to prove is that \emph{there never are more than $N$ balls on the screen at the same time}. Since we know that we generate a ball at each \statement{creationInterval} seconds, this property is equivalent to verify that, $N \times \statement{creationInterval}$ seconds after the generation of a ball, such ball has already exited from the screen. In fact, we know that after such time ($N \times \statement{creationInterval}$ seconds), other $N$ balls will have been generated, so the first ball must have already exited. \footnote{This property is more restrictive than the original one, since the balls do not necessarily exit the screen in the same order they were created, depending on their random initial velocity. Anyway, the increased strictness means that, if we prove this property, we have also proven the original one.}

The theoretical interest of this case study lies in the fact that non-relational or non-disjunctive approaches are not properly suited to verify \emph{Property 1}. Consider for example the Interval domain where every variable of the program is associated to a single interval. After a few iterations, when the vertical position possibly goes to zero, the analysis is no more able to distinguish which branch of the \statement{if-then-else} to take. In this case, the lub operator makes the vertical velocity interval quite wider, since it will contain both positive and negative values. After that, the precision gets completely lost, since the velocity variable affects the position and vice-versa. On the other hand, the accuracy that would be ensured by using existing disjunctive domains has a computational cost that makes this approach unfeasible for practical use.

