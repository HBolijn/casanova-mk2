%%%%%%%%%%%%%%%%%%%%%%%%%%%%%%%%%%%%%%%%%%%%%%%%%%%%%%%%%%
% script_monad_case_study.tex
%%%%%%%%%%%%%%%%%%%%%%%%%%%%%%%%%%%%%%%%%%%%%%%%%%%%%%%%%%

\subsubsection{A sample script}

The state of the game contains a series of star systems, fleets, players and various other information:

\begin{lstlisting}
type GameState = {
    StarSystems : StarSystem List
    Fleets      : Fleet List
    ... }
\end{lstlisting}

We define a series of accessor functions that allow a coroutine to access the current state of the game; such an accessor function, for example, is the \texttt{get\_fleets} that returns the list of active fleets:

\begin{lstlisting}
let get_fleets = 
  script{
    let! state = get_state
    return state.Fleets }
\end{lstlisting}

The basic mode of the game uses our scripting system to determine the winner of the game; as long as there is more than one player standing, the script waits. This script computes the union of the set of active fleet owners with the set of system owners:  

\begin{lstlisting}
let alive_players_set = 
  script{
    let! fs = get_fleets
    let fleet_owners = 
             fs |> Seq.map (fun f -> f.Owner) 
             |> Set.ofSeq
    let! ss = get_systems
    let system_owners = 
             ss |> Seq.map (fun s -> s.Owner) 
             |> Set.ofSeq
    return fleet_owners + system_owners }

let game_over =
  script{
    let! alive_players = alive_players_set
    let num_alive_players = alive_players |> Seq.length
    return num_alive_players = 1 }
\end{lstlisting}

The main task of our script is to wait until the set of active players has exactly one element; when this happens, that player is returned as the winner: 

\begin{lstlisting}
let basic_game_mode = wait_game_over game_over
\end{lstlisting}

The two short snippets above are all there is to the main game mode.

Variations of the game are soccer (one system acts as the ball which can be moved around), capture the flag, siege and others. We omit a detailed discussion of the other variations for reasons of space; the important thing to realize is that all of these variations have been implemented with the same simplicity of the scripts above, by instancing one game pattern with appropriate scripts which are built with a mix of combinators interspersed with accesses to the game state.

\subsubsection{Input Management}

Another large subsystem where we have used our scripting system is input management. Input is divided into a series of pairs of scripts; each pair of scripts is separated by a guard: the first script performs an event detection, while the second performs an event response. Each pair of scripts is repeated forever, in parallel with all other scripts.

As an example, consider the following script that decides whether to launch ships or not against a target:
\begin{lstlisting}
let input world = 
  repeat_(
  script{ 
    if mouse_clicked_left() then
      let mouse = mouse_position()
      // find closest planet under the cursor
      let clicked:Option<Planet> = ... 
      return Some(clicked) } => 
    fun p -> script{ world.SourcePlanet := Some(p) }) .||
  repeat_(
  script{ 
    if mouse_clicked_right() && 
       world.SourcePlanet <> None then
      let mouse = mouse_position()
      // find closest planet under the cursor
      let clicked:Option<Planet> = ... 
      match clicked with 
      | Some planet -> 
        return Some(planet,world.SourcePlanet.Value)
      | None -> return None } => 
    fun (source,target) -> 
      script{ mk_fleet world source target })
\end{lstlisting}

A distinct advantage of this technique is that it allows us to cleanly separate the code that reads the actual user input from the code that performs something meaningful on the game world with this input. By parametrizing the code above with respect to the input detection scripts we could make it possible to support different controller types (game pad, touch panel, mouse, keyboard, etc.).

\subsubsection{Further uses}

Since our scripting system proved effective in the various areas where we tried it, and given that its performance is fully satisfactory, we have decided to use it more pervasively all over the game. The result is that each entity (planets, ships, players, etc.) has a large chunk of its game logic moved from the update loop into appropriate scripts.

Our menu system is heavily based on our scripts (this proved especially useful when implementing the multi-player lobby, where the regular menu (buttons, input detection, etc.) had to be run interleaved with the scripts that synchronize the list of players across the network.

Finally, we have used scripts for the entire networking system, given that many operations such as connections and time-outs require timers and many parallel operations.