%%%%%%%%%%%%%%%%%%%%%%%%%%%%%%%%%%%%%%%%%%%%%%%%%%%%%%%%%%
% intro.tex
%%%%%%%%%%%%%%%%%%%%%%%%%%%%%%%%%%%%%%%%%%%%%%%%%%%%%%%%%%

Computer games promise to be the next frontier in entertainment, with
game sales having topped movie and music sales in 2010 \cite{ESA}. 
This unprecedented market prospects and potential for computer-game 
diffusion among end-users has created substantial interest 
in research on principled design techniques and on 
cost-effective development technologies for game architectures. 
Our present endeavour makes a step along these directions. 

The architecture of a computer game is centered around the game's
underlying discrete simulation engine  \cite{DESIGN_PC_GAME_ENGINE}, 
a complex piece of software which provides the core functionalities
related to the {\em game content} -- most notably the game logic,
determining the behavior of the various game characters -- 
and at the same time acts as the central coordination unit among the
different components of the architecture -- the rendering and graphic
algorithms and  functions, the physics engine, the
input-output/networking devices, $\dots$ and so on. 

Given their inherent complexity, game engines are usually written in
low-level languages in order to ensure the high performance figures
required for a smooth playing experience. Unfortunately, the use of
such languages has two main drawbacks. First, game logic and
AI require very frequent changes and fine tuning throughout all
phases of game development. As such, coding gameplay functionalities
within the engine  implies long (re)compilation times and hence heavy
design/development costs. Secondly, game logic and AI  
are built, modified and maintained by game designers,  who are often
rather inexperienced with computer programming concepts and
techniques and hence may hardly operate effectively with low-level
programming languages. 

To counter these shortcomings, modern game architectures tend
to support a clean separation between the development of the game
engine and its content, with engines made \textit{scriptable}, so that
their gameplay functionalities can be programmed without direct access
to the engine internals and with a higher-level language. This is
typically achieved by embedding into the game engine the virtual
machine of the chosen language and by exporting the engine
itself as an API for that language. Scripting then provides the
ability to code major game functionalities directly within code
snippets (the  scripts) that are interpreted by the game engine,
without touching  the engine at all. This makes games highly
customizable and at the  
same time keeps the development costs under control by drastically
reducing compilation time and greatly simplifying the scripting experience. 
Games scripting has become common practice, to the point that most 
commercial games nowadays come integrated either with off-the-shelf scripting 
languages, most notably LUA \cite{SCRIPTING_LUA}, Python \cite{SCRIPTING_PYTHON} or
C\# \cite{UNITY_YIELD}, or else with  in-house built
languages \cite{UNREALSCRIPT_LATENT_FUNCTIONS}.   

In the present paper, we propose a novel, statically typed game
scripting language based on a monadic DSL \cite{MONADIC_DSL} built 
on top of F\# \cite{FRIENDLY_FSHARP}. Our DSL combines the flexibility of programming
abstractions comparable to those offered  by LUA with the benefits of
strong, static typing. In addition, our DSL language supports a rather
smooth integration between the execution model of the simulation
engine, based on discrete-time updates of the game state, and the
logic implemented  by the scripts, which typically encode actions that
span multiple update time-slots. As in other scripting languages, this
integration   is achieved by equipping our DSL with coroutines, which
we encode  within the monadic operators for binding and return. As a
result our DSL offers greater flexibility over coroutines which 
are wired inside the virtual machine itself; this flexibility, 
in turn, makes it possible to tailor our scripting system precisely
around the requirements of the game, without knowledge about the (complex)
internals of a virtual machine. Also, encapsulating coroutines inside
a monad effectively makes them mostly transparent to the developer.
Interestingly, the additional flexibility comes with very limited
overhead: indeed, our scripts run faster than LUA's and at least as
fast as C\# scripts. We have tested extensively our scripting system by
using it in many projects, among which the most important is arguably
\cite{GALAXY_WARS}, an upcoming strategy game that uses our scripts for
managing most of its game logic, animating the world entities, handling
user input and even implementing the networking subsystem. Our scripting 
system also lends itself to be programmed through a visual editor for
non-technical users, such as gameplay designers and others.

\paragraph{Paper structure} Section \ref{sec:scripting_in_games} gives a
brief review of game architectures and of the technical issues arising
in scriptable engines. Section \ref{sec:script_monad} details the core 
runtime of our DSL, while Section \ref{sec:script_combinators}
develops the DSL itself as a library of combinators built
around the core that support a powerful set of customized behaviors,
and illustrates an actual example of our DSL at work in the
development of a game. Section \ref{sec:benchmarks} presents our
results, Section \ref{sec:visual_editor} discusses the introduction of a 
visual environment to create scripts for non-technical users and Section 
\ref{sec:conclusions} concludes the presentation.    




