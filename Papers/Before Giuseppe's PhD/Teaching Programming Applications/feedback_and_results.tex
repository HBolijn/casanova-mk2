%%%%%%%%%%%%%%%%%%%%%%%%%%%%%%%%%%%%%%%%%%%%%%%%%%%%%%%%%%
% feedback_and_results.tex
%%%%%%%%%%%%%%%%%%%%%%%%%%%%%%%%%%%%%%%%%%%%%%%%%%%%%%%%%%

We have tried to assess the results of both programs according to the following indicators:

\begin{itemize}
\item perceived satisfaction
\item timely completion of the exercises
\item increase in enrollment
\end{itemize}

Given our original purpose of stimulating interest in Computer Science we are mostly interested in the first item, student satisfaction. Given that we also wished to offer students an active and engaging experience, we also set out to monitor how many students completed our exercises and to what degree. Finally, we wish to monitor the effectiveness of our approach in getting more students enrolled in our degree program.

Unfortunately, we still do not have data about the effectiveness of this year's initiative in terms of student enrollment, given that the next enrollment period has not started yet. We can report as preliminary results the fact that this year we witnessed a small increase in enrollment and we had experimented for the first time with an initiative that was similar to this one (indeed, last year we tried what was the pilot program for this year's initiative). Hopefully this trend will be confirmed in a few months, and in a few years we will have enough data to be of statistical significance.

\paragraph{Computer Graphics}

The computer graphics results have been quite good. As mentioned previously all the students completed their assignments in time. Being the computer graphics assignments both complex \textit{and} time consuming, we were very surprised that all the students managed to work their way through all exercises, even the harder ones. Indeed, we had a ``plan B'' that consisted in showing the correct results after each exercise if too many students had not been able to complete it, in order to level the class so that it could keep moving at a steady pace; such plan was never needed!

The number of students that participated in this course was 31. A questionnaire was created that contained 8 true/false questions about the topics; a few sample questions (we do not include them all for reasons of space) were:

\begin{itemize}
\item ``normals must be transformed into world space before lighting''
\item ``Phong lighting is more accurate than Lambert lighting but is more computationally expensive''
\item ``texture filtering is needed for sampling in-between texels''
\end{itemize}

The resulting error percentages are:

\begin{table}[htb]
\centering
\begin{tabular}{|c|c|c|c|c|c|c|c|}
\hline
Q1 & Q2 & Q3 & Q4 & Q5 & Q6 & Q7 & Q8 \\
\hline
13\% &	0\%	& 7\%	& 3\%	& 7\%	& 3\%	& 7\%	& 20\% \\
\hline
\end{tabular}
\caption{Comprehensions test}
\end{table}

As we can easily see, on average questions scored more than 90\% right answers. The students' perceived satisfaction was measured with another questionnaire, which asked questions about each their experience:

\begin{table}[htb]
\centering
\begin{tabular}{|p{1.2cm}|p{1.2cm}|p{1.2cm}|p{1.2cm}|p{1.2cm}|}
\hline
Glad of having participated & Clarity of teachers & Teaching method & Useful-ness of notions & General satisfaction 		\\
\hline
83\%	& 64\%	& 75\%	& 74\%	& 75\% \\
\hline
\end{tabular}
\caption{Student satisfaction}
\end{table}


Students are very satisfied of having participated in the experience, even though they found the frontal parts of the lesson too hard. In general the results are good, and we believe that by simplifying a bit the exercises and by spending more time on the background (vectors and trigonometry) these scores could become higher. It is interesting to notice that we also received enthusiastic comments from some students, who wrote further feedback such as \textit{An amazing experience: it shows an aspect of Computer Science I did not believe existed}; many students observed that through this kind of course they came to understand that Computer Science is richer and more complex than they believed.


\paragraph{Physics Simulation}

In a manner similar to that used with the computer graphics short-course we have tried assessing the results of the second short-course. In this case we only administered a final questionnaire about the students satisfaction, and not about their understanding of the matters taught. All students completed all their tasks, and in \textit{far less time} than we had anticipated.

11 students participated in this course. They reported their satisfaction as:

\begin{table}[htb]
\centering
\begin{tabular}{|c|c|c|c|c|}
\hline
Interesting & Fun & Engaging & Boring & Uninteresting\\
\hline
82\% & 36\% & 27\% & 0\% & 0\% \\
\hline
\end{tabular}
\caption{General satisfaction}
\end{table}

\begin{table}[htb]
\centering
\begin{tabular}{|p{1.2cm}|p{1.2cm}|p{1.2cm}|p{1.2cm}|p{1.2cm}|}
\hline
Too hard & Hard & Ok & Easy & Too easy \\
\hline
0\% & 9\% & 82\% & 0\% & 0\% \\
\hline
\end{tabular}
\caption{Difficulty}
\end{table}

\begin{table}[htb]
\centering
\begin{tabular}{|p{1.4cm}|p{1.4cm}|p{1.4cm}|p{1.4cm}|}
\hline
New concepts & Concepts I already knew & New way to study concepts I already knew & Nothing new for me \\
\hline
64\% & 0\% & 36\% & 0\% \\
\hline
\end{tabular}
\caption{Novelty}
\end{table}

As we can see, this second course obtained very high ratings. All students described the course as either fun, interesting, engaging; all but one student found the difficulty appropriate (neither too easy nor too hard); finally, all students felt that we taught them something new, either new concepts or new approaches to concepts they knew already.

In addition two students wrote that they really enjoyed the introductory portions which had been somewhat lacking in the computer graphics short-course and that they really liked the opportunity to experiment theoretical concepts in a programming setting.

As a final notice, we wish to express satisfaction from the fact that we were able to teach differential equations without the usual level of students' stress. Differential equations are often considered a benchmark of obscurity, difficulty and boredom, and we are glad we were able to show them as beautiful mathematical concepts which elegantly capture the essence of important real-world phenomena.

