%%%%%%%%%%%%%%%%%%%%%%%%%%%%%%%%%%%%%%%%%%%%%%%%%%%%%%%%%%
% computer_graphics.tex
%%%%%%%%%%%%%%%%%%%%%%%%%%%%%%%%%%%%%%%%%%%%%%%%%%%%%%%%%%

The main objective of the Computer Graphics short-course is to show students how to compute a realistic appearance for some mesh (3D geometry composed of triangles) through shaders that simulate the visual properties of different materials.

The students are given a starting shader, which is a small program that is run by the GPU and which is responsible for computing the position of the vertices of each triangle and for computing the color of each pixel of the various triangles.

The general shape of a shader is the following (PARAMETERS, VERTEX\_SHADER and PIXEL\_SHADER are just place holders for other code):

\begin{lstlisting}
let shader = 
<@@
  PARAMETERS

  VERTEX_SHADER
  
  PIXEL_SHADER
  
  in ()
@@>
\end{lstlisting}

The shader parameters are a set of global variables that are set to the GPU memory and which store properties of the scene such as the position and color of lights, transformation matrices, etc. The simplest parameters for a 3D scene are provided in the initial shader, and are three transformation matrices: 

\begin{lstlisting}
let World         = parameter() : Matrix
let View          = parameter() : Matrix
let Projection    = parameter() : Matrix
\end{lstlisting}

The students are also given a list of all the global variables that are supported by our system.

A vertex shader takes as input a vertex (which stores the position of the vertex plus optional additional attributes such as its color, its normal, etc.) and returns the transformed vertex (which again contains at least the transformed position plus optional additional attributes). The sample vertex shader simply transforms the input position from 3D space to screen space:

\begin{lstlisting}
  let vertex_shader (InputPosition(pos)) =
    let worldPosition = pos * World
    let viewPosition  = worldPosition * 
                        View
    let pos'          = viewPosition * 
                        Projection

    in OutputPosition(pos')
\end{lstlisting}

A pixel shader takes as input a subset of the attributes of a transformed vertex and returns the color of its pixels; the sample pixel shader returns a uniform red color:

\begin{lstlisting}
  let pixel_shader () =
    OutputColor(Vector4(1f,0f,0f,1f)) 
\end{lstlisting}

Running the initial configuration simply draws a red mesh. Starting from this the students add the normal to the vertex shader input and compute first Lambert lighting and then Phong lighting with a specular component. At this point a color texture is added and read with the input texture coordinates found in the vertex shader; by combining the texturing shader and the Phong shader results a shader capable of showing fine details and lighting (Figure 1). At this point the students replace the background texture with a cubemap which gives the illusion of a more detailed background. The final results are obtained by computing the reflected view direction with respect to the normal of the mesh in order to look up the background, thereby giving the illusion of a reflective surface; by perturbing the normals of the surface in an animated fashion the last shader can be used to show a very pleasant-looking reflective water surface (Figure 2).
