%%%%%%%%%%%%%%%%%%%%%%%%%%%%%%%%%%%%%%%%%%%%%%%%%%%%%%%%%%
% solution_workflow.tex
%%%%%%%%%%%%%%%%%%%%%%%%%%%%%%%%%%%%%%%%%%%%%%%%%%%%%%%%%%

\begin{figure}
\begin{center}
\includegraphics[scale=0.6]{Solution_workflow.png}
\end{center}
\label{fig:solution_workflow}
\caption{Solution workflow}
\end{figure}

In Figure \ref{fig:solution_workflow} we can see a diagram depicting the steps used by our system when processing an X3D scene (plus its accompanying scripts). In the figure red blocks represent data while the blue blocks represent computations.
We start with an X3D file which describes our scene. This file may contain some scripts as routes, in \texttt{script} node, or the scripts may be stored into an external file. There are two layers of transformations described by our system:

\begin{itemize}
\addtolength{\itemsep}{-0.5\baselineskip}
\item a transformation from the original external scripts into our F\# scripts
\item a transformation from the original entities and routes of the X3D file into the final program
\end{itemize}

In the current development stage we have implemented part of the first transformation (we translate those scripts that are expressed as routes) and the second transformation (we can process any scene entity). The X3D scene and its routes are translated into F\# source code. This source code contains a type definition that describes the entire scene and the routes, plus an update function that translates the activation of scripts as a consequence of the changes in the entities that result from a user action or from the temporal evolution of the scene. Such conversion therefor supports the translation of regular X3D nodes that describe shapes and the various scene entities, and also routing nodes that describe basic scripts.

Our system supports \textit{external scripts}, that is any script that is not expressed as a route, but only if they are provided already translated to F\#; this means that our system cannot translate and process scripts written in Javascript and Java, but it can process those scripts if the translation has been done by hand. External scripts in F\# are validated against our type definitions, to ensure that they correctly access the scene. If their validation succeeeds, the final program is produced that integrates both the scene and all the scripts. This second processing gives our system support for any other scripts which are not easily expressed with routes.

We have used F\# \cite{FRIENDLY_FSHARP,FSHARP}, a multi-paradigm functional programming language targeting the .NET Framework. It is a variant of ML and is largely compatible with the OCaml implementation. F\# enjoys full support in the .NET Framework, meaning that it can take advantage of all .Net libraries (such as XNA for game development, which is most useful to us) and powerful IDEs such as Visual Studio and MonoDevelop.