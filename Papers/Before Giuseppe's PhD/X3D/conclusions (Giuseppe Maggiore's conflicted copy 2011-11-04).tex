%%%%%%%%%%%%%%%%%%%%%%%%%%%%%%%%%%%%%%%%%%%%%%%%%%%%%%%%%%
% conclusions.tex
%%%%%%%%%%%%%%%%%%%%%%%%%%%%%%%%%%%%%%%%%%%%%%%%%%%%%%%%%%

In this paper we have presented a novel approach to optimizing X3D scenes. Upon recognition that X3D requires the highest possible degree of performance and safety we have experimented with a move from the slower, dynamic interpretation that current X3D browsers do to a faster, static compiled model of execution which creates a ``specialized browser'' for every X3D scene.

Creating complex applications with X3D alone is not possible, and the included scripting solutions do not scale (in the experience of the authors) to domains such as video games without a lot of effort. For this reason we have studied a way to better integrate and validate scripts into X3D scenes; our solution of embedding monadic F\# scripts into the compiled code allows a developer to add complex logic to a script in a seamless manner, since the syntax tree that contains our compiled scene can be further manipulated to include any scripts we want without the overhead of casting and dynamic dispatching. Thanks to quotations we are sure that the compiled result is valid (routes are correct, etc.) and scripts correctly access the scene nodes; any error will be detected at compile time, thus reducing the amount of testing needed by the developers.

Thanks to our system it has been possible to run our compiled X3D scenes with different platforms that support XNA. In particular we have tested our benchmark scenes on the Xbox 360 and Windows Phone 7. While the Xbox is very similar to a PC in terms of hardware, the ability of running X3D scenes on powerful mobile devices is extremely interesting since it unlocks new interaction opportunities; moreover, optimizations such as ours become crucial to make good use of the limited computing power of such devices.

Our work is by no means complete. We still need to implement some of the primitives of our target X3D profile (the \textit{Interactive} profile). Also, we are planning to extensively test our system for games and interactive applications with a very complex logic. Also, we wish to experiment a further expansion of our compiler to support customized \textit{computations layers} such as custom physics, custom AI, custom renderers that perform visibility culling or advanced shading such as ray-tracing; our aim is to make X3D more suitable for game development and richer, applications. Finally, we wish to carefully implement all these aspects of interactive applications so that fast and high-quality execution on mobile devices remains possible.
