\documentclass{sig-alternate}

\usepackage{amssymb}
\usepackage{amsmath}
\usepackage{graphicx}
\usepackage{epsfig}
\usepackage{subfigure}
\usepackage{listings}
\usepackage{natbib}
\setlength{\bibsep}{0.0pt}
\usepackage{verbatim}
\usepackage[T1]{fontenc} 
\usepackage[hyphens]{url}
\lstset{language=ml}
\lstset{commentstyle=\textit}
\lstset{mathescape=true}
\lstset{backgroundcolor=,rulecolor=}
\lstset{frame=single}
\lstset{breaklines=true}
\lstset{basicstyle=\scriptsize\ttfamily}

\pdfpagewidth=8.5in
\pdfpageheight=11in

\begin{document}

\title{A compilation technique to increase X3D performance and safety}

\numberofauthors{4}
\author{
Giuseppe Maggiore \and Fabio Pittarello \and Michele Bugliesi \and Mohamed Abbadi\\
       \affaddr{Universit\`a Ca' Foscari Venezia}\\
       \affaddr{Dipartimento di Scienze Ambientali,}\\
       \affaddr{Informatica e Statistica}\\
       \email{\{maggiore,pitt,michele,mabbadi\}@dais.unive.it}
}

\date{}

\maketitle

\begin{abstract}
As virtual worlds grow more and more complex, virtual reality browsers and engines face growing challenges. These challenges are centered on performance on one hand (an interactive framerate is always required) and complexity on the other hand (the larger and more articulated a virtual world, the more immersive the experience).

The usual implementation of an engine or browser for running virtual worlds features an object-oriented architecture of classes. This architecture is a source of often underestimated overhead in terms of dynamic dispatching, and dynamic lookups by scripts when they try to access portions of the scene are both costly and possible sources of mistakes.

In this paper we discuss how we have tackled the problem of increasing performance in X3D browsers while also making scripts safe. We have used a compilation technique that removes some overhead and which allows us to introduce safety for scripts that access the state
\end{abstract}

\category{D.1.1}{Programming Techniques}{Applicative (Functional) Programming} 
\category{D.2.2}{Soft\-ware Engineering}{Software Libraries}[Design Tools and Techniques]
\category{D.2.13}{Soft\-ware Engineering}{Reusable Software}[Domain engineering, Reusable libraries, Reuse models]
\category{D.3.3}{Programming Languages}{Language Constructs and Features}
\category{D.3.4}{Pro\-gramming Languages}{Processors}[Optimization, Run-time environments]
\category{H.5.1}{Information Systems}{Information Interfaces and Presentation}[Multimedia Information Systems]

\terms{Performance,Reliability,Languages}

\keywords{x3d, performance, safety, compilation}

\section{Introduction}
\label{sec:intro}
%%%%%%%%%%%%%%%%%%%%%%%%%%%%%%%%%%%%%%%%%%%%%%%%%%%%%%%%%%
% intro.tex
%%%%%%%%%%%%%%%%%%%%%%%%%%%%%%%%%%%%%%%%%%%%%%%%%%%%%%%%%%

Game are:
- complex
- require performance

Low-level languages do not cut it fully.

We will:
(A) discuss game constraints
  - discuss traditional approaches and their limitations
  - discuss similarities between game genres and try and find a general framework
  - define a language that is built around this general framework, and which makes it easy
  - reason on why we'd rather have a new language and not just a library
  - give syntax, typing, semantics
(B) give optimization transformations of the language
(C) give a detailed case study
  - give benchmarks that show how effective our optimizations are and how they are completely automated (they require no effort on the part of the programmer) 

\section{Solution Workflow}
\label{sec:solution_workflow}
%%%%%%%%%%%%%%%%%%%%%%%%%%%%%%%%%%%%%%%%%%%%%%%%%%%%%%%%%%
% solution_workflow.tex
%%%%%%%%%%%%%%%%%%%%%%%%%%%%%%%%%%%%%%%%%%%%%%%%%%%%%%%%%%

\begin{figure}
\begin{center}
\includegraphics[scale=0.6]{Solution_workflow.png}
\end{center}
\label{fig:solution_workflow}
\caption{Solution workflow}
\end{figure}

In Figure \ref{fig:solution_workflow} we can see a diagram depicting the steps used by our system when processing an X3D scene (plus its accompanying scripts). In the figure red blocks represent data while the blue blocks represent computations.
We start with an X3D file which describes our scene. This file may contain some scripts as routes, in \texttt{script} node, or the scripts may be stored into an external file. There are two layers of transformations described by our system:

\begin{itemize}
\addtolength{\itemsep}{-0.5\baselineskip}
\item a transformation from the original external scripts into our F\# scripts
\item a transformation from the original entities and routes of the X3D file into the final program
\end{itemize}

In the current development stage we have implemented part of the first transformation (we translate those scripts that are expressed as routes) and the second transformation (we can process any scene entity). The X3D scene and its routes are translated into F\# source code. This source code contains a type definition that describes the entire scene and the routes, plus an update function that translates the activation of scripts as a consequence of the changes in the entities that result from a user action or from the temporal evolution of the scene. Such conversion therefor supports the translation of regular X3D nodes that describe shapes and the various scene entities, and also routing nodes that describe basic scripts.

Our system supports \textit{external scripts}, that is any script that is not expressed as a route, but only if they are provided already translated to F\#; this means that our system cannot translate and process scripts written in Javascript and Java, but it can process those scripts if the translation has been done by hand. External scripts in F\# are validated against our type definitions, to ensure that they correctly access the scene. If their validation succeeeds, the final program is produced that integrates both the scene and all the scripts. This second processing gives our system support for any other scripts which are not easily expressed with routes.

We have used F\# \cite{FRIENDLY_FSHARP,FSHARP}, a multi-paradigm functional programming language targeting the .NET Framework. It is a variant of ML and is largely compatible with the OCaml implementation. F\# enjoys full support in the .NET Framework, meaning that it can take advantage of all .Net libraries (such as XNA for game development, which is most useful to us) and powerful IDEs such as Visual Studio and MonoDevelop.

\section{Compiling The Scene}
\label{sec:compiling_scene}
%%%%%%%%%%%%%%%%%%%%%%%%%%%%%%%%%%%%%%%%%%%%%%%%%%%%%%%%%%
% compiling_scene.tex
%%%%%%%%%%%%%%%%%%%%%%%%%%%%%%%%%%%%%%%%%%%%%%%%%%%%%%%%%%

In this section we show an outline of our compilation technique. 

The first step our compiler performs is deserializing the xml definition of our X3D scene. The scene is then processed and turned into a record, a type definition that describeds the static structure of our scene. The record contains:

\begin{itemize}
\addtolength{\itemsep}{-0.5\baselineskip}
\item a field for each static node of the scene; each field has the name of the node if the node has a \texttt{DEF} attribute
\item a field for a list of dynamic nodes
\item a field for a list of active scripts
\end{itemize}

A sample state for a scene with a timer and a box could be:

\begin{lstlisting}
type Scene  =
  {
    myClock       : Timer 
    box           : Box
    dynamic_nodes : List<Node>
    script        : Script
  }
\end{lstlisting}

Where \texttt{Timer} and \texttt{Box} are the concrete classes for a timer and a box respectively, and they both inherit from the \texttt{Node} class.
A list of nodes is needed to represent the dynamic portions of the scene, and a list of scripts maintains the sequence of currently running scripts.

This state definition is quite important, since it represents the interface between our scene and our scripts, and since it allows us fast lookups of specific nodes. Finding a node now just requires reading from a field in the state, an operation which is both fast and certain not to fail. For example, looking for the \texttt{time} field of the \texttt{"myClock"} node would simply require writing:

\begin{lstlisting}
scene.myClock.time
\end{lstlisting}

We then proceed to the initialization of the state. This amounts to creating instances of each node, and then assigning these instances to the fields of the \texttt{scene} variable.

An \texttt{update} function is then constructed that performs the update of all the statically known fields of the state, and which also executes the various routes of the scene. Also, the \texttt{update} function invokes the (dynamically dispatched) \texttt{update} function of each dynamic node; this is necessary because it would be unrealistic to hope that a complex virtual world can exclusively rely on statically known nodes, and a balance must be struck between optimizing static nodes and supporting dynamic ones.

The \texttt{update} function also performs a tick for all currently running scripts.

The update function that updates the state seen above would simply become:

\begin{lstlisting}
let update (dt:float32) =
  scene.myClock.update dt
  scene.box.update dt
  for node in scene.dynamic_nodes do
    node.update dt
  scene.script.update dt
\end{lstlisting}

We could add a simple route that moves the \texttt{box} node along the Y-axis according to the current time of the \texttt{myClock} node by adding the following line of code to the \texttt{update} function:

\begin{lstlisting}
  scene.box.Position.Y <- scene.myClock.Time
\end{lstlisting}

In general, routes are simple assignments when translated into our system.

\section{A More Detailed Example}
\label{sec:case_study}
%%%%%%%%%%%%%%%%%%%%%%%%%%%%%%%%%%%%%%%%%%%%%%%%%%%%%%%%%%
% case_study.tex
%%%%%%%%%%%%%%%%%%%%%%%%%%%%%%%%%%%%%%%%%%%%%%%%%%%%%%%%%%

We will now present a more detailed example to see our compiler in action by showing how it handles all the features of an X3D scene: entities and routes. We will consider an X3D scene that contains a looping timer which updates a color that in turn updates the material used when drawing a box:

\begin{lstlisting}[language=xml]
<Scene>
  <ColorInterpolator DEF='myColor'
    keyValue='1 0 0, 0 1 0, 0 0 1, 1 0 0'
    key='0.0 0.333 0.666 1.0'/>
  <TimeSensor DEF='myClock' cycleInterval='10.0' loop='true'/>
  <Shape>
    <Box/>
    <Appearance>
      <Material DEF='myMaterial'/>
    </Appearance>
  </Shape>
  <ROUTE fromNode='myClock' fromField='fraction_changed'
         toNode='myColor' toField='set_fraction'/>
  <ROUTE fromNode='myColor' fromField='value_changed'
         toNode='myMaterial' toField='diffuseColor'/>
</Scene>
\end{lstlisting}

Our compiler produces the following state definition from the above scene:

\begin{lstlisting}
type Scene =
  {
    myColor       : ColorInterpolator
    myClock       : TimeSensor
    myMaterial    : Material
    dynamic_nodes : List<Node>
    script        : Script
  }
\end{lstlisting}

where pointers to all statically known nodes are maintained.

The initialization function for our state initializes a set of local variables, one for each named node, and then builds the actual scene state. Notice that at this point routes are ignored, since they will be used only for the update function:

\begin{lstlisting}
let scene = 
  let myColor = 
       ColorInterpolator(
         keyValue = [ ... ],
         key = [ ... ])
  let myClock = 
       TimeSensor(
         cycleInterval = 10.0,
         loop = true)
  let myMaterial = Material()
  let dynamic_nodes = 
        [
          Shape(
            Value = 
              Box(Appearance(Value = myMaterial)))
        ]
  {
    myColor        = myColor
    myClock        = myClock
    myMaterial     = myMaterial
    dynamic_nodes  = dynamic_nodes
    script         = null
  }         
\end{lstlisting}

After initializing the scene without a script, we can load the script from an external parameter that will be assigned in the linking phase. Loading a script requires passing to it the scene, so that the script may access the scene to manipulate it:

\begin{lstlisting}
scene.script := load_script scene
\end{lstlisting}

The update function invokes the internal update function of all nodes, starting from the statically known and ending with the dynamic ones. Routes are executed in the update function:

\begin{lstlisting}
let update dt = 
  scene.myClock.update dt
  scene.myColor.update dt
  scene.myMaterial.update dt
  for node in scene.dynamic_nodes do
    node.update dt
  scene.script.update dt
  
  myColor.fraction <- myClock.fraction
  myMaterial.diffuseColor <- myColor.value
\end{lstlisting}

It is important to notice that routes in the update function are represented by the actual chains of field updates that need to be performed; there is no overhead when dynamically propagating the update events. Also, if a field does not start a route then there are no ``hidden'' costs as we would have when firing a \texttt{FieldModified} event with no routes listening.


\section{Benchmarks}
\label{sec:benchmarks}
%%%%%%%%%%%%%%%%%%%%%%%%%%%%%%%%%%%%%%%%%%%%%
% BENCHMARKS
%%%%%%%%%%%%%%%%%%%%%%%%%%%%%%%%%%%%%%%%%%%%%

- Windows, Xbox, Wp7 (, iPad?)
- memory recycling
- parallel execution
- query optimization
 

\section{External Scripts}
\label{sec:compiling_scripts}
%%%%%%%%%%%%%%%%%%%%%%%%%%%%%%%%%%%%%%%%%%%%%%%%%%%%%%%%%%
% compiling_scripts.tex
%%%%%%%%%%%%%%%%%%%%%%%%%%%%%%%%%%%%%%%%%%%%%%%%%%%%%%%%%%

Scripting is a very important part of game development \citep{BETTER_SCRIPTS_GAMES}. For this reason we have adapted to our system a scripting solution that is derived from game engines. Whereas many game engines either use Lua, Python or even C\# as scripting languages (with various advantages and disadvantages) \cite{SCRIPTING_LUA,SCRIPTING_PYTHON, UNITY_YIELD} we have used F\# which we believe offers a powerful mix of the best features of all these languages: coroutines, flexibility and a lightweight syntax make F\# scripts similar to LUA and Python while static typing and support for .Net libraries and IDEs put F\# on par with C\# in terms of broader support.

To give additional expressive power to our scene, we add support to external scripts; external scripts are all those scripts that cannot be expressed in terms of routes. External scripts are very general, that is they can perform complex data conversions when copying values across entities, and they may even create, remove and modify nodes in any way possible. To represent external scripts, rather than using arbitrary objects that can access the state we have chosen to use coroutines, a widely used mechanism for representing computations in interactive applications \cite{PYTHON_COROUTINES,GPU_GEMS_6}. Coroutines are subroutines that can be suspended and resumed at certain locations. With coroutines the code for a SM is written ``linearly'' one statement after another, but each action may suspend itself (an operation often called ``yield'') many times before completing. A coroutine stores a temporary, internal state transparently inside its continuation.

We build a monadic framework \cite{COMPR_MON,DECL_IMP,EFF_MON,MOGGI_MON} for coroutines that allows us greater customization flexibility. This way we can define our own system for combining scripts running them in parallel, concurrently, etc. For a detailed discussion of this monadic framework for scripts and coroutines see \cite{X3D_TR1}.

A script in our system is defined as a normal F\# program surrounded by \texttt{\{ \}} brackets. A script runs another script with the statements \texttt{let!} and \texttt{do!}, and scripts can be combined with a small set of operators.

The main operators to combine scripts are:
\begin{itemize}
\item \texttt{parallel} ($s_1 \wedge s_2$) executes two scripts in parallel and returns both results
\item \texttt{concurrent} ($s_1 \vee s_2$) executes two scripts concurrently and returns the result of the first to terminate
\item \texttt{guard} ($s_1 \Rightarrow s_2$) executes and returns the result of a script only when another script evaluates to \texttt{true}
\item \texttt{repeat} ($\uparrow s$) keeps executing a script over and over
\end{itemize}

A sample script that moves the box \texttt{myBox} when the user enters a certain region \texttt{myRegion} could be the following:

\begin{lstlisting}
let my_script (scene:Scene) =
  let rec animate =
    script {
      if scene.myBox.Position.Y < 100.0f then
        scene.myBox.Position.Y <- scene.myBox.Position.Y + 0.1f
        do! animate }
  script {
    do! guard
         script {
           return inside(scene.Camera.Position, myRegion) }
         animate }
\end{lstlisting}

Notice that our script has a parameter of type \texttt{Scene}. If this parameter is used incorrectly (for example the scene this script is applied to does not have a \texttt{Box} node with name \texttt{myBox}) we will get a compile-time error. This makes it easier to build larger, reusable script modules since a mistake in using a pre-made module is easier to spot and requires less testing. Using scripts which have been made for different scenes would require extensive testing to ensure at least that all node accesses are correct.

Our scripting system is expressive enough to represent many scripts running together, even if at a first glance it may appear that our system supports only a single script. By using the \texttt{parallel} operator we can combine together a large number of scripts. For example, let us say we have many scripts $s_1, ... , s_n$ that must all run together with our scene. Each scripts has a different duration, that is the not all scripts will end at the same time (indeed, a script may even run indefinitely). The main script would chain each of the various actual scripts in the following manner:

\begin{lstlisting}
let my_script (scene:Scene) =
  parallel $s_1$ (parallel $s_2$ ... $s_n$) ... )
\end{lstlisting}
 

\section{Conclusions and future work}
\label{sec:conclusions}
%% Changed by PS, April 4, 2014.

\section{Future work}
\label{sec:future_work}
The Casanova 2 language is capable of implementing usable and quite complex games. The language, while usable, is currently still in development as it misses a few features. In particular, support for multiplayer games is at this moment lacking. We believe that the existing mechanisms for handling time offered by Casanova 2 could be augmented with relatively little effort in order to greatly simplify the hard task of building multiplayer games. This is part of future work, that we are currently engaging in. We are also doing usability studies using students from various disciplines and backgrounds.

The high level view of the game that the Casanova 2 compiler provides can be exploited in order to improve the programmer experience. This means that we could use tools for code analysis (such as abstract interpretation \cite{nielson1999principles} or type system extensions) in order to better understand the game being built, and to help with correctness analysis, performance analysis, or even optimization.


%\subsection{User study}
%We wish to perform an in-depth user study for Casanova 2 to improve usability in the development process. We have already performed a partial (and quite promising) small user study which we will extend and complete.


%We have performed the following test: we gathered a group of students of game programming and a group of students of game design. We gave them a series of Casanova 2 samples, printed on paper. Each student had to guess the functionality of each sample, and sketch a screen-shot. Furthermore, each student also provided some additional feedback on the language.

%The samples were: (\textit{i}) a string of text moved around the screen with the keyboard, (\textit{ii}) a string of text that moves along a predefined path automatically, and (\textit{iii}) an asteroid shooter.

%Eleven (over a total of thirteen) students understood the samples completely, both drawing the screen-shots and explaining the dynamics of the game correctly. Two students were lost on the syntactic differences between Casanova 2 and the more familiar C-like syntax. The direct feedback was mostly centred around a series of common observations, which are reported in Table \ref{students_feedback}. For each observation, the table reports how many times we encountered it.

%\begin{table}[!t]
%% increase table row spacing, adjust to taste
%\renewcommand{\arraystretch}{1.3}
% if using array.sty, it might be a good idea to tweak the value of
% \extrarowheight as needed to properly center the text within the cells

%\caption{Feedback from students}
%\label{students_feedback}
%\centering

%% Some packages, such as MDW tools, offer better commands for making tables
%% than the plain LaTeX2e tabular which is used here.
%\begin{tabular}{|c||c|}
%\hline
%Syntax is unfamiliar at first & 3\\
%\hline
%Syntax is clear & 8\\
%\hline
%Indentation instead of parentheses is a downside & 2\\
%\hline
%List processing with queries is very effective & 1\\
%\hline
%Rules are a good abstraction for games & 2\\
%\hline
%\end{tabular}
%\end{table}

%We also built a significantly bigger sample, which we asked only three students to study. The sample is a checkpoint-based RTS (see Figure \ref{RTS game} for a screenshot). All students correctly identified the game mechanics, and provided some additional feedback. Most of this feedback overlaps with that obtained for the samples, but some new observations emerge. Arguably, some patterns become visible only with larger samples:
%\begin{itemize}
%\item \texttt{wait} and \texttt{when} are very powerful
%\item Multiple rules on the same field are very powerful
%\item Multiple rules on the same field may lead to behaviours that are complex to understand
%\end{itemize}


\section{Conclusions}
\label{sec:conclusions}

Casanova 2, a language specifically designed for building computer games, may offer a solution for the high development costs of games. The goal of Casanova 2 is to reduce the effort and complexities associated with building games. Casanova 2 manages the game world through entities and rules, and offers constructs (wait and yield) to deal with the run-time dynamics. As shown by the benchmarks in Section \ref{sec:evaluation}, we believe that we have taken a significant step towards reaching these goals. In fact, we achieved at the same time very good performance and simplicity, thereby empowering developers with limited resources.  

\bibliographystyle{plain}
\bibliography{references} 

%\cite{*}
\nocite{}

\end{document}
