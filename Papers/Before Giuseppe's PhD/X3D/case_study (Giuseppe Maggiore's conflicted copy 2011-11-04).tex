%%%%%%%%%%%%%%%%%%%%%%%%%%%%%%%%%%%%%%%%%%%%%%%%%%%%%%%%%%
% case_study.tex
%%%%%%%%%%%%%%%%%%%%%%%%%%%%%%%%%%%%%%%%%%%%%%%%%%%%%%%%%%

We will now present a simple case study to see our compiler in action. We will consider an X3D scene that contains a looping timer which updates a color that in turn updates the material used when drawing a box:

\begin{lstlisting}[language=xml]
<Scene>
  <ColorInterpolator DEF='myColor'
    keyValue='1 0 0, 0 1 0, 0 0 1, 1 0 0'
    key='0.0 0.333 0.666 1.0'/>
  <TimeSensor DEF='myClock' cycleInterval='10.0' loop='true'/>
  <Shape>
    <Box/>
    <Appearance>
      <Material DEF='myMaterial'/>
    </Appearance>
  </Shape>
  <ROUTE fromNode='myClock' fromField='fraction_changed'
         toNode='myColor' toField='set_fraction'/>
  <ROUTE fromNode='myColor' fromField='value_changed'
         toNode='myMaterial' toField='diffuseColor'/>
</Scene>
\end{lstlisting}

We do not want to interpret this scene dynamically into a browser. Rather, we want to compile this scene into a specialized browser which has the above scene \textbf{hardcoded} inside its source.

The source code that implements the above scene is the following:

\begin{lstlisting}
let rec myScene () = 
    
  let rec Appearence1 = 
    new Appearance(myMaterial)
  and myMaterial = 
    new Material("myMaterial",
      new Vector3(0.80f,0.80f,0.80f))
  and Box1 = 
    new Box("Box1",myMaterial,Vector3.One)
  let rec Shape1 = 
    new Shape(Box1,Appearence1)
  and myColor = 
    new ColorInterpolator("myColor",
      [
        new Vector3(1.00f,0.00f,0.00f);
        new Vector3(0.00f,1.00f,0.00f);
        new Vector3(0.00f,0.00f,1.00f);
        new Vector3(1.00f,0.00f,0.00f)],
      [0.00f;0.33f;0.67f;1.00f])
  and myClock = 
    new TimeSensor(true,10.00f)
        
  and load_content = 
    Box1.set_Model( game.Content.Load( "Box"))
    
  and update dt = 
    myClock.time <- (myClock.time + dt)
    myColor.fraction <- myClock.fraction
    myMaterial.diffuseColor <- myColor.value
        
    if (myClock.time > 10.00f) then
      myClock.time <- 0.00f
      myColor.fraction <- myClock.fraction
      myMaterial.diffuseColor <- myColor.value
    
  and draw dt = 
    Box1.Draw(dt)

  in { update = update; 
        draw = draw; 
        load_content = load_content; })
\end{lstlisting}

Notice that we need to create all the nodes, unfolding them so that even if there are nested DEFs (such as the one for \texttt{myMaterial}) they appear as top level identifiers accessible to the rest of the program. This way we have completed our mapping from named nodes to local \texttt{let}-bindings, so that accessing the nodes of our scene will correspond to the common, simple and fast variable lookups.

We also need to define three functions: an initialization function, which loads any required data from disk; an update function, which performs the game logic and executes the routes; a draw function, which draws the scene to the screen.

Notice that routes in the update function are represented by the actual chains of field updates that we need to happen; there is no overhead at all when dynamically propagating the update events. Also, if a field does not start a route then there are no ``hidden'' costs as we would have when firing a \texttt{FieldModified} event with no listeners.


