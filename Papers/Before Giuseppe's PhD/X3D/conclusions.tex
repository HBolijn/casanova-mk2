%%%%%%%%%%%%%%%%%%%%%%%%%%%%%%%%%%%%%%%%%%%%%%%%%%%%%%%%%%
% conclusions.tex
%%%%%%%%%%%%%%%%%%%%%%%%%%%%%%%%%%%%%%%%%%%%%%%%%%%%%%%%%%

In this paper we have presented a novel approach to optimizing X3D scenes. Upon recognition that X3D requires the highest possible degree of performance and safety we have experimented with a move from the slower, dynamic interpretation that current X3D browsers do to a faster, static compiled model of execution which creates a ``specialized browser'' for every X3D scene.

Thanks to type safety we are sure that the compiled result is valid (routes are correct, etc.) and that external scripts correctly access the scene nodes; any error will be detected at compile time, thus reducing the amount of testing needed and of errors ending up in the final release.

Thanks to our system it has been possible to run our compiled X3D scenes with different platforms that support XNA. In particular we have tested our benchmark scenes on the Xbox 360 and Windows Phone 7. While the Xbox is very similar to a PC in terms of hardware, the ability of running X3D scenes on powerful mobile devices is extremely interesting since it unlocks new interaction opportunities; moreover, optimizations such as ours become crucial to make good use of the limited computing power of such devices.

Our work is by no means complete. We still need to implement some of the primitives of our target X3D profile (the \textit{Interactive} profile). We would like to experiment with automated compilation of Javascript and Java scripts, to increase our support of X3D in this direction as well. Also, we wish to study if there are other possible classes of optimizations that can be performed during our code generation phase. Finally, we wish to study what kind of optimizations are needed to ensure fast and high-quality execution of interactive virtual worlds on mobile devices.
