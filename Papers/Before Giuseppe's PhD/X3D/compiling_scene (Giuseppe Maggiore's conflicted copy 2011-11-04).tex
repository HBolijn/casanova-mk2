%%%%%%%%%%%%%%%%%%%%%%%%%%%%%%%%%%%%%%%%%%%%%%%%%%%%%%%%%%
% compiling_scene.tex
%%%%%%%%%%%%%%%%%%%%%%%%%%%%%%%%%%%%%%%%%%%%%%%%%%%%%%%%%%

The first step our compiler performs is deserializing the xml definition of our X3D scene.

The scene is then processed and turned into a record, a type definition that describeds the static structure of our scene. The record contains:

\begin{itemize}
\item a field for each static node of the scene; each field has the name of the node if the node has a \texttt{DEF} attribute
\item a field for a list of dynamic nodes
\item a field for a list of active scripts
\end{itemize}

A sample state for a scene with a timer and a box could be:

\begin{lstlisting}
type SceneState =
  {
    myClock       : Timer 
    box           : Box
    dynamic_nodes : List<Node>
    script        : Script
  }
\end{lstlisting}

Where \texttt{Timer} and \texttt{Box} are the concrete classes for a timer and a box respectively, and they both inherit from the \texttt{Node} class.
A list of nodes is needed to represent the dynamic portions of the scene, and a list of scripts maintains the sequence of currently running scripts.

This state definition is quite important, since it represents the interface between our scene and our scripts, and since it allows us fast lookups of specific nodes. Finding a node now just requires reading from a field in the state, an operation which is both fast and certain not to fail. For example, looking for the \texttt{time} field of the \texttt{"myClock"} node would simply require writing:

\begin{lstlisting}
state.myClock.time
\end{lstlisting}

We then proceed to the initialization of the state. This amounts to creating instances of each node, and then assigning these instances to the fields of the \texttt{state} variable.

An \texttt{update} function is then constructed that performs the update of all the statically known fields of the state, and which also executes the various routes of the scene. Also, the \texttt{update} function invokes the (dynamically dispatched) \texttt{update} function of each dynamic node; this is necessary because it would be unrealistic to hope that a complex virtual world can exclusively rely on statically known nodes, and a balance must be struck between optimizing static nodes and supporting dynamic ones.

The \texttt{update} function also performs a tick for all currently running scripts.

The update function that updates the state seen above would simply become:

\begin{lstlisting}
let update (dt:float32) =
  state.myClock.update dt
  state.box.update dt
  for node in state.dynamic_nodes do
    node.update dt
  script.update dt
\end{lstlisting}
