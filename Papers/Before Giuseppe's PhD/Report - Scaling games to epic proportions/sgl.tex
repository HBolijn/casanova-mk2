%
% sgl.tex
%

Game data is modeled as a relation $E$. This table is a multiset and it needs not have keys. Each row in the table represents a unit or object, with information such as health, speed, damage, special properties, etc. Each row also contains data representing messages to this unit, such as data coming from the pathfinding system, cooldown periods, and so on. One possible definition of $E$ for our example could be:

\begin{lstlisting}
E(key,player,posx,posy,health,cooldown,
  weaponused,movevect_x,movevect_y,damage,inaura)
\end{lstlisting}

The attributes \texttt{key ... cooldown} represent the state of the unit; these attributes are not modified directly from a script, but rather they are modified accordingly to the values of \texttt{weaponused ... inaura}, which represent the effects that unit is being subjected to.

An SGL script consists of a single action for a single unit, and at each tick it will be executed and it will take an entire environment $E$ and it will return a new environment $E_u$. All environments $E_u$ are then combined into a single environment in which the effects are applied with a post-processing step. A possible post-processing related to the schema above could be (where \texttt{norm} normalizes movement velocity):

\begin{lstlisting}
SELECT u.key, u.player, 
       u.posx + u.movevect_x * norm AS posx, 
       u.posy + u.movevect_y * norm AS posy, 
       u.health - u.damage + u.inaura AS health, 
       u.cooldown - 1 + u.weaponused*_TIME_RELOAD AS cooldown, 
       0 AS weaponused, 
       0 AS movevect_x, 
       0 AS movevect_y, 
       0 AS damage, 
       0 AS inaura
FROM E u 
WHERE u.health > 0
\end{lstlisting}

The post-processing step above applies the effects to the state and resets the effects to avoid applying them again during the next time step.

\paragraph{Syntax of SGL}
SGL scripts have a simple syntax consisting of a mix of ML (\texttt{let, if-then-else}) and SQL:

\begin{lstlisting}
main(u) { 
  (let c = CountEnemiesInRange(u,u.range)) 
  (let away_vector = (u.posx, u.posy) -
         CentroidOfEnemyUnits(u, u.range)) { 
           if (c > u.morale) then
             perform MoveInDirection(u,away_vector); 
           else if (c > 0 and u.cooldown = 0) then
             (let target_key = getNearestEnemy(u).key) { 
             perform FireAt(u, target_key);
} } }
\end{lstlisting}

Where the auxiliary functions described above can be defined as:

\begin{lstlisting}
function CountEnemiesInRange(u, range) 
returns SELECT Count( * ) 
        FROM	E
        WHERE E.x >= u.posx - range
        AND	E.x <= u.posx + range
        AND	E.y >= u.posy - range
        AND	E.y <= u.posy + range
        AND	E.player <> u.player;
\end{lstlisting}

In general, actions have the following syntax:

\begin{lstlisting}
action ::= (let attributename = term) 
         | action action; action 
         | if cond then action 
         | if cond then action else action 
         | perform action_name
\end{lstlisting}


\paragraph{Combining effects}
Effects in an environment $E$ must be combined somehow. To do so, we add annotations to each attribute in the environment schema:

$$E(K,A_1:\tau_1,...,A_n:\tau_n)$$

where $\tau_i$ is an aggregate function such as \texttt{min, max, avg, sum, ...} which defines the way multiple effects on each unit are combined. A combination operator $\oplus R$ is defined as:

\begin{lstlisting}
select K, $f_{i_1}$($A_{i_1}$) as $A_{i_1}$,...,$f_{i_m}$($A_{i_m}$) as $A_{i_m}$
from R group by K,$A_{i_1}$,...,$A_{i_l}$;
\end{lstlisting}

where $f_i(A_i)$ is defined as $A_i$ if $\tau_i = const$ (that is no aggregation is needed on attribute $i$) while it is defined as $\tau_i$ otherwise (that is it uses the aggregate function described by $\tau_i$ itself). We can define (unambiguously) another meaning for $\oplus$: $R \oplus S = \oplus(R \biguplus S)$ where $\biguplus$ denotes multiset union.

In the case of the schema described above, the $\oplus E$ environment can be computed as:

\begin{lstlisting}
SELECT key, player, posx, posy, health, cooldown, 
       max(weaponused) AS weaponused, 
       sum(movevect_x) AS movevect_x, 
       sum(movevect_y) AS movevect_y, 
       sum(damage) as damage,
       max(inaura) as inaura
FROM E
GROUP BY key, player, posx, posy, health, cooldown
\end{lstlisting}

Each action can be seen as a function from a unit and the environment into the environment which at each evaluation step applies the combination operator; actions have type:

$action : Env \times Multiset(Env) \rightarrow Multiset(Env)$

which is the denotational equivalent of a stateful function. The first parameter is the tuple representing the current unit, the second parameter is the set of all units, the result is the new set of all units. The semantics of SGL actions can be given in terms of a semantic function $\llbracket \bullet \rrbracket$:

\begin{lstlisting}
$\llbracket$let v = t in f$\rrbracket$(u,E) 	:= $\llbracket$f[v $\mapsto$ $\llbracket$ t $\rrbracket$](u,E)$\rrbracket$(u,E)
$\llbracket$f;g$\rrbracket$(u,E)             	:= $\llbracket$f$\rrbracket$(u,E) $\oplus$ $\llbracket$g$\rrbracket$(u,E)
$\llbracket$if c then f$\rrbracket$(u,E)     	:= if $\llbracket$c$\rrbracket_{cond}$(u,E) then
$\llbracket$perform(G)$\rrbracket$(u,E)     	:= $\llbracket$G$\rrbracket$(u,E)
\end{lstlisting}

where \texttt{G} is a built-in action function or a user-defined auxiliary function, and where \texttt{if-then-else} can easily be defined in terms of sequential conditionals with the same condition (negated in the \texttt{else} branch).

\paragraph{Optimization}
As with typical relational systems, there are two possible optimization venues: algebraic and physical. Algebraic optimizations are applied to queries in order to ensure that a script combines as often as possible in order to keep the number of elements to process to the minimum possible; for example, rather than compute:

$$\oplus(R \biguplus S)$$

it is usually convenient to compute:

$$\oplus(\oplus R \biguplus \oplus S)$$

thanks to the fact that the $\oplus$ operator returns a smaller set (effects are condensed together into the state).

More important in terms of performance gain is the use of indices.

Sometimes a unit must process an aggregate function for example to count the number of nearby units; for example, a unit may wish to flee if its allies are overwhelmed by a large enemy force. This script can be na\"ively computed with an $O(n^2)$ algorithm. This computation can be greatly accelerated with the introduction of an index for the aggregate that allows to share the results of the aggregation across several units.

SGL queries have the advantage that the set of queries is fixed a priori, and thus the indices can be tailored around each individual query plan. Also, it is important to realize that while indices can be used to share information between nearby units, indices are discarded and rebuilt at the beginning of each clock tick beacuse of the very high dynamicity of unit data even across one single tick; for the most important and often updated indices such as those on unit positions, it is more efficient to do so than it is to maintain a dynamic index.

The use of indices reduces the complexity of many common AI algorithms from $O(n^2)$ to $O(n \log n)$. This gives games the possibility to scale to an order of magnitude more units without incurring in a significant performance hit.