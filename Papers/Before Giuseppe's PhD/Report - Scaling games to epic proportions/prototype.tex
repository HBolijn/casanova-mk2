%%%%%%%%%%%%%%%%%%%%%%%%%%%%%%%%%%%%%%%%%%%%%%%%%%%%%%%%%%
% prototype.tex
%%%%%%%%%%%%%%%%%%%%%%%%%%%%%%%%%%%%%%%%%%%%%%%%%%%%%%%%%%

\paragraph{Overview}

Fast prototyping of video-games:

\begin{itemize}
\item define state
\item ``guess'' (with the help of annotations) draw and update functions
\item offer multi-threaded, safe custom logic via coroutines
\end{itemize}


State -> State\_i
We start by giving a definition of the game state $\sigma$. From this state we compute (with a type-to-type function) the state of a slice of the state which represents the read-only portion of the state which can be safely read concurrently and the writable portion of the state which can be written exclusively by a single thread.

\begin{lstlisting}
type Ship =
  {  
     Position:Ref Vector3;
     Velocity:Ref_norm Vector3
     Damage:float
     EnginePower:float
     Mass:float
     Life : Ref_sum float;
     Damage : float
     Model : Model
     Source:System
     Destination:System
     Owner : Player
   }
   
type System =
  {
    Position : Vector3
    ParkedShips : Ref_sum float
    Attackers : Ref_sum [Ref Ship]
    ProductionRate : float
    Defence : float
    Neighbors : [System]
    Owner : Ref Player
  }

type Player =
  {
    OwnedSystems : Ref [Ref System]
    OwnedShips : Ref [Ref Ship]
  }    

type State = 
   {  
     Systems : [System]
     MovingShips : Ref_sum [Ref Ship]
     Players : [Player]
   }
\end{lstlisting}


State -> update
At this point we give a definition of the simplest possible game-logic that can be guessed from the game state; this definition implements simple looping, collision detection, etc. based on declarative annotations that express simple relationships between portions of the state.

\begin{lstlisting}
position ship = integral(ship.Velocity,dt)
velocity ship = integral(ship.Acceleration,dt)
acceleration ship =
  let a = Vector3() * ship.EnginePower / ship.Mass
  if dist(ship.Position,ship.Source.Position) <
     dist(ship.Position,ship.Destination.Position) then
    integral(a,dt)
  else
    integral(-a,dt)
\end{lstlisting}

\begin{lstlisting}
owned_systems player = state.Systems |> filter (.Owner = player)
owned_ships player = state.MovingShips |> filter (.Owner = player)
\end{lstlisting}

\begin{lstlisting}
damage system =
  integral(system.AttackingShips 
               |> filter (dist(.Position,.Target.Position) < .Range)
               |> sum (.Damage),dt)
parked_ships system =
  integral(system.ProductionRate+damage system,dt)
owner system = 
  if system.ParkedShips <= 0.0 and system.AttackingShips <> [] then
    system.AttackingShips.[0].Owner
  else
    system.Owner
\end{lstlisting}


State -> draw 
After generating the update code, we must generate the drawing code. The drawing code is the simplest code that is capable of rendering the game state entities based on further declarative annotations.

\begin{lstlisting}
appearance ship = ship.Position, ship.Model
appearance system = 
  system.Position, Model("star.fbx") +
  Text("system_font.font",system.ParkedShips |> floor),
  [for n in system.Neighbors =>
     Texture("link.png",Link(system.Position,n.Position)]
\end{lstlisting}

Note that the generated application is still not a game: there is no input management and no interaction with the user is happening. The results of launching the game are simply the correct updates that in our example animate the moving ships from their source system to their destination system. There is also no AI.


Custom code:
At this point we need to inject code into the generated application in order to be able to implement customized behaviours. The user provides a set of functions (responsible of managing a portion of the game state) which are automatically run and mapped on various threads:

\begin{lstlisting}
type Yield = () -> ()
type Quit = () -> ()
type Control = { yield : Yield; quit : Quit }
run_state : State -> [(State_i -> Control -> ())] -> ()
\end{lstlisting}

\texttt{run\_state} takes as input the initial state, a list of functions that take a slice of the state and a record with various control functions and runs an update loop and \texttt{run\_state} creates a set of threads (not necessarily one for each function of the list) and runs the update functions over these threads.

\begin{lstlisting}
run_state : {...} [...; ...; ...]
\end{lstlisting}


Coroutines:
To further simplify our job we offer support for coroutines in order to simplify the implementation of state machines (which are ubiquitous in games: a very large percentage [] of every games' entities use state machines).

\begin{lstlisting}
Script
\end{lstlisting}


\paragraph{Language}

Syntax:

\begin{lstlisting}
t ::= let, fun, app, pair, record, if, let rec, ...
        | t := t
        | !t
        | let!
        | return
        | mk_ref t
\end{lstlisting}


Types:

\begin{lstlisting}
T ::= unit | int | string | ...
        | T * T
        | T + T
        | { l : T; l : T } ({l1 : T1; ...; ln : Tn} is a shortcut for {l1:T1; {l2:T2;{...ln:Tn}...}})
        | Ref_agg T
        | Ref_agg_i T where agg = max, min, sum, prod, avg, ...
\end{lstlisting}


Type functions:

\begin{lstlisting}
State_i T = Everywhere (Ref_agg T) -> Ref_agg_i T 
\end{lstlisting}
