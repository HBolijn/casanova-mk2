%
% conclusions.tex
%

Innovations in game architecture are often responsible for pushing further the boundaries of game design. By building more powerful \textit{and} more scriptable game engines it is possible at the same time to create games where game logic is faster and more complex, resulting in a more compelling experience for the player.

Building a system that allows for a smarter, individualized AI could usher an era of games that offer a smarter experience where units and non-playing characters make smart choices rather than waiting for ``obvious'' instructions from the player.

It must also be noted that games have been historically responsible for many evolutions of modern personal computing, by pushing the hardware to its limits and by offering powerful visual experiences that years or decades later become paradigms in non-gaming software. For this reason this kind of language evolution must be put in its right context: it is not just a way to make better games, but it is also a prototype for a way to write better \textit{programs} in general.

\paragraph{Future research directions in the field}
From a survey of the current panorama of game development tools and languages it appears that a technological shift is happening: from the almost exclusive use of C and C++ for creating games, plus the odd use of Lua or Python as scripting languages, more and more games are being made in modern managed languages such as C\# or Objective-C. The introduction of XNA (a framework that allows independent game developers to publish commercial games on the XBox 360 without costly licenses) has marked the beginning of the adoption of C\#-based frameworks for making games; other similar frameworks have been used in games between a C++ engine and the scripting system. This shift signals that the game development industry may be starting to have trouble with its old infrastructure given the rising development costs, the shrinking budgets and the decreasing development time and it is looking into new languages with more abstraction to increase its efficiency. For this reason many researchers (the authors of this report being part of this group) are studying new ways to leverage the power coming from the field of functional and declarative languages in order to more easily automate some repetitive programming tasks that often require complex and cumbersome libraries.