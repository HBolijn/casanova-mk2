%----------------------------------------------------------------------------
%  intro.tex 
%----------------------------------------------------------------------------
Our goal is to define a system for managing memory safely
and without leaks. We also strive for automated memory deallocation through the use of smart pointers rather than
garbage collection for performance reasons. The kind of 
program that we will use as our main benchmark is a simple
game where the player must avoid being hit by a certain
number of asteroids. In this simple game our entities (the
list of asteroids) are continuously allocated and
deallocated:

\begin{lstlisting}

type Vector2 = (Int,Int)
type Ship = Vector2
type Asteroid = Vector2

// library functions for accessing vectors
// ...

type State = Playing (Ship, [Asteroid], Int) | Won | Lost

// initialize the player ship
state := Playing(State(Ship(0,0), [], 0))

update dt = 
  case state of
  | Playing(ship,asteroids,score) ->
    if Input.MoveLeft then
		ship.X := ship.X - dt
    if Input.MoveRight then
		ship.X := ship.X + dt
    foreach a in asteroids
      if a.Y > Screen.Height then
        asteroids.delete a
        score := score + dt
      if intersects a ship then 
        state := Lose
      a.Y := a.Y + dt
    if asteroids.Count < CreateThreshold then
      asteroids.new(Asteroid(rand(0,Screen.Width),0))
    if score > MaxScore then 
      state := Win
  | _ -> ()
\end{lstlisting}

The $update$ function will be invoked very often; indeed,
to achieve an interactive frame-rate for the simulation
$update$ must be called at least 20-30 times per second.
Even more depending on how powerful the host CPU is. The
$dt$ parameter is a floating point value that represents
how long it has been since the last invocation of the
function and is used for numerically integrating the
physics and AI equations of the simulation.

We will achieve our objective gradually. First we will
give a formal definition of the source language and its
type system. Then we will define a completely static
system for managing memory that implements just a portion
of our source language; this first system will be written
in Haskell, which is an excellent tool for type-safe
proofs of concept given its hugely expressive type system. Then we will extend this system with dynamic capabilities,
at the same time migrating  our code to the F\# language which offers a more appropriate benchmark suite for performance. Also, given its vast set of real-world
libraries such as XNA [XNA] for game development and given
the language capabilities for implicit mutability allows 
us for some greater flexibility in the implementation of
the internals of our system. Using F\# comes at a cost:
the language does not have the same powerful type system
as Haskell does, so we will have to use explicit
meta-programming features such as quotations [QUOTATIONS]
and reflection [REFLECTION]. Finally we will add reference
counting to remove the burden of freeing memory from the
shoulders of the developer.


As a side note, our system offers two unintended, yet
beneficial aspects. The first aspect is that memory is
managed explicitly through a library; this makes some 
powerful extensions almost free to add. We discuss one
of these that enables lock-less [LOCKLESS] concurrent 
programming by using the type of the state of mutable
computations to regulate accesses. Various other possible
extensions are discussed in [FUTURE WORK]. The second
beneficial aspect is that all our entities fields are
accessed through explicitly computed labels which can
easily be passed around:

\begin{lstlisting}
move_right entity position =
  entity.position.X := entity.position.X + 1
\end{lstlisting}

making it possible to implement self-describing objects
[SELF REP] and some form of type-safe reflection.


Our main contributions are the following:
\begin{itemize}
\item to remove the need for a garbage collector in a 
      high-level language to the great benefit of
      performance;
\item to use phantom types to regulate memory accesses
      for safety and more.
\end{itemize}