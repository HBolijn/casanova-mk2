We now implement a simple example that shows how we can define a system of mutable vectors.

We begin by defining a 2d vector ($Vector2$) with two methods and a 3d vector ($Vector3$) with two other methods and which inherits the 2d vector:
\begin{lstlisting}
type Vector2Def k = Field St Float `Malloc` Field St Float `Malloc` Method St k () () `Malloc` Method St k () Float `Malloc` Nil
data RecVector2 = RecVector2 (Vector2Def RecVector2)
type Vector2 = Vector2Def RecVector2
instance Recursive Vector2 where
  type Rec Vector2 = RecVector2
  to = RecVector2
  from (RecVector2 v) = v
x :: Label St Vector2 (Field St Float)
x = labelAt Z
y :: Label St Vector2 (Field St Float)
y = labelAt (S Z)
norm2 :: Label St Vector2 (Method St (Rec Vector2) () ())
norm2 = labelAt (S (S Z))
len2 :: Label St Vector2 (Method St (Rec Vector2) () Float)
len2 = labelAt (S (S (S Z)))

mk_vector2 :: Float -> Float -> Vector2
mk_vector2 xv yv = (StField xv) `Malloc` (StField yv) `Malloc` norm `Malloc` len `Malloc` Nil
                    where norm = mk_method (\this -> \() -> do l <- (this <= len2) ()
                                                               (this <= x) *= (/ l)
                                                               (this <= y) *= (/ l))
                          len = mk_method (\this -> \() -> do xv <- eval (this <= x)
                                                              yv <- eval (this <= y)
                                                              return (sqrt(xv * xv + yv * yv)))

type Vector3Def k = Inherit Vector2 `Malloc` Field St Float `Malloc` Method St k () () `Malloc` Method St k () Float `Malloc` Nil
data RecVector3 = RecVector3 (Vector3Def RecVector3)
type Vector3 = Vector3Def RecVector3
instance Recursive Vector3 where
  type Rec Vector3 = RecVector3
  to = RecVector3
  from (RecVector3 v) = v
z :: Label St Vector3 (Field St Float)
z = labelAt (S Z)
norm3 :: Label St Vector3 (Method St (Rec Vector3) () ())
norm3 = labelAt (S (S Z))
len3 :: Label St Vector3 (Method St (Rec Vector3) () Float)
len3 = labelAt (S (S (S Z)))

mk_vector3 :: Float -> Float -> Float -> Vector3
mk_vector3 xv yv zv = StInherit (mk_vector2 xv yv) `Malloc` StField zv `Malloc` norm `Malloc` len `Malloc` Nil
                      where norm = mk_method (\this -> \() -> do l <- (this <= len3) ()
                                                                 ((get_base this) <= x) *= (/l)
                                                                 ((get_base this) <= y) *= (/l)
                                                                 (this <= z) *= (/l)) 
                            len = mk_method (\this -> \() -> do xv <- eval ((get_base this) <= x)
                                                                yv <- eval ((get_base this) <= y)
                                                                zv <- eval (this <= z)
                                                                return (sqrt(xv * xv + yv * yv + zv * zv)))

ex4 :: forall mem . mem ~ (Vector3 `Malloc` Nil) => St Nil Nil Bool
ex4 = do+ v <- new (mk_vector3 0.0 2.0 (-1.0))
          xv <- eval ((get_base v) <= x)
          let v' = coerce v :: Ref St mem Vector2
          xv' <- eval (v' <= x)
          (v' <= norm2)()
          (v <= norm3)()
          return (xv == xv')
          delete

res4 :: Bool
res4 = runSt ex4 Nil
\end{lstlisting}

where we expect that $res=True$. Notice how select labels that are defined for a 2d vector on an instance of a 3d vector, and how we access the same label on the value of $base$ obtained through coercion on the 3d vector.
